\documentclass{article}
%\input{extdef}
\newcommand{\ve}[1]{\mbox{\boldmath ${#1}$}}
\begin{document}

\begin{center}
 \textbf{ Parcial de An\'alisis Matem\'atico IV (Hilbert-Fourier) \\
Noviembre 19, 1999 }
\end{center}

Problema 1.- 
%
Sea $f(x) = |x|$ en el intervalo $[-\pi,\pi]$. 

a) Vea que esta funci\'on es Lipshitz.

b) Calcule sus coeficientes de Fourier.

c) Use la convergencia punto a punto (no la relaci\'on de parseval) para concluir que 

\begin{equation}
  \frac{\pi^2}{8} = \sum_{k=1}^{\infty} \frac{1}{(2k+1)^2}
\end{equation}

Problema 2.-
%
Encuentre los coeficientes de Fourier de la funci\'on $|\sin(x)|$.

Problema 3.-
%
Prueve que los coeficientes de Fourier de $h(x) = f(x)g(x)$
estan dados por 
\begin{equation}
h_n = \frac{1}{\sqrt{2\pi}} \sum_{m=-\infty}^{\infty} f_{n-m} g_m.
\end{equation}





Problema 4.-
%
Verdadero o Falso? Justifique.

\begin{itemize}

\item[a] 
La transformada parcial de una funci\'on Lipshitz $f$, $S_N(f)$
aproxima puntualmente y uniformemente a $f$ cuando $N\to \infty$.

\item[b] 
Si una funci\'on es cont\'\i{}nuamente diferenciable ($f \in C^1([0,2\pi])$),
luego la mejor cota para sus coeficientes de Fourier es: $|f_n| < C/\sqrt{n}$,
donde $C$ depende solo $f$ y su derivada primera.

\item[c] 
Las funciones cont\'\i{}nuas no son densas en $L^2$.

\item[d] 
En un espacio de Hilbert puede haber sucesiones de Cauchy que no
convergen a ning\'un elemento del mismo.

\item[e] 
El espacio 
$H := \{f: R \to R | f \in C^0(R), \|f\|_H < \infty \}$, 
con norma dada por
$\|f\|_H := \sqrt{\int_{-\infty}^{\infty} |f(x)|^2 dx}$
es un espacio de Hilbert. 
\end{itemize}






\end{document}

%%% Local Variables: 
%%% mode: latex
%%% TeX-master: t
%%% End: 
