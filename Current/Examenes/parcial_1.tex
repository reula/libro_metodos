\documentclass{article}
%\input{extdef}
\newcommand{\ve}[1]{\mbox{\boldmath ${#1}$}}
\begin{document}

\begin{center}
 \textbf{ Parcial de An\'alisis Matem\'atico IV (ODEs) \\
Noviembre 9, 1999 }
\end{center}

Problema 1.- Escribir la matriz $A$ y el vector $\vec{x}$
en la base en la cual $A$ es diagonal.


\begin{equation}
  A := \left( 
             \begin{array}{ccccc}
               0 & -i& 0 & 0 & 0 \\
               i & 0 & 0 & 0 & 0 \\
               0 & 0 & 3 & 0 & 0 \\
               0 & 0 & 0 & 1 & -i \\
               0 & 0 & 0 & i & 1
             \end{array}
             \right) 
\;\;\;\;\;\;
 \vec{x} := \left( 
             \begin{array}{c}
               0  \\
               a  \\
               i  \\
               b  \\    
               -1 
             \end{array}
         \right).
\end{equation}


Problema 2.-
%
Dos cuerpos de masa $m_1$ y $m_2$ estan unidos entre si por un resorte , 
de constante $k_2$, mientras que el primero est\'a unido a una pared por
otro resorte de constante $k_1$.

Las ecuaciones del sistema son:
\begin{eqnarray}
  m_1\frac{d^2x_1}{dt^2} &=& -k_1 x_1 + k_2 x_2  \\
  m_2\frac{d^2x_2}{dt^2} &=& -k_2 x_2
\end{eqnarray}
Lleve a un sistema de primer orden y encuentre su soluci\'on general.
Descr\'\i{}bala usando el operador $\ve{X}^t_{t_0}$ dando sus
componentes en la base en la que queda el sistema de primer orden.

Problema 3.-
%
Encuentre las soluciones estacionarias del siguiente sistema y determine si son
estables o no.

\begin{eqnarray}
  \dot{x} &=& x - x^3 - xy^2 \\
  \dot{y} &=& 2y - y ^5 -x^4y
\end{eqnarray}

Problema 4.-
%
Resuelva por el m\'etodo de variaci\'on de constantes la siguiente 
ecuaci\'on:

\begin{equation}
\frac{dx}{dt} + x \cos(t) = \frac{1}{2} \sin(2t).
\end{equation}




%Sea una masa $m$ conectada a un resorte de constante $k_$ y con constante 
%de disipaci\'on $\gamma$ sobre la que actua una fuerza 
%$f(t) = F\cos(\omega t)$.
%La ecuaci\'on de evoluci\'on es: 
%$m\frac{d^2 x}{dt^2} = -kx -\gamma \frac{dx}{dt} + F\cos(\omega t)$.
%Encuentre la soluci\'on general.


Problema 5.-
Verdadero o Falso? Justifique.

\begin{itemize}

\item[a] Sea $\ve{A}$ un operador cuya representaci\'on matricial en una dada base tiene forma triangular superior y en ella sus componentes a lo largo de la
diagonal son distintas [$A^i{}_i \neq A^j{}_j $ si $i\neq j$]. Luego el operador es 
diagonalizable, es decir existe una base en la cual los \'unicos elementos
distintos de cero son los diagonales.

\item[b] Sea $V$ el espacio vectorial de polinomios de grado igual o menor
que $n$. Sea $\ve{D}$ el operador lineal derivada, es decir 
$\ve{D} P(x) := \frac{dP(x)}{dx}$. Luego el \'unico autovalor de $\ve{D}$ es cero.

\item[c] La ecuaci\'on $\frac{dx}{dt} = 2 + \sqrt{x}$ tiene m\'as de una soluci\'on para cierto dato inicial.

\item[d] Sea $f: R \to R$ diferenciable y tal que existen constantes
positivas $A$ y $B$ tales que $|f(x)| < A + B|x|$. Luego cualquier soluci\'on
de $\frac{dx}{dt} = f(x)$ existe para todo $t$.

\item[e] La ecuaci\'on 
\begin{equation}
\frac{d^5x}{dt^5} + 3 \frac{d^3x}{dt^3} + 9x =0
\end{equation}
tiene $5$ soluciones linealmente independientes.

\end{itemize}






\end{document}

%%% Local Variables: 
%%% mode: latex
%%% TeX-master: t
%%% End: 
