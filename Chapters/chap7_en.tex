\chapter{Proof of the Fundamental Theorem}

We will only prove points $i)$ and $ii)$, for which we will use the
method of successive approximations by Picard, which is important in applications. The proof of point $iii)$ is of a technical nature and does not provide any additional significant insight. Point $iv)$ follows from points $i)$ and $ii)$, and its proof is identical to that of Corollary 1.2 of the analogous Theorem for the case of an ODE.

To prove these points, we will need to develop some ideas and results from the mathematical theory of infinite-dimensional vector spaces. We will restrict ourselves to the minimum required by the theorem since these topics will be developed more extensively in the second part of this course.

\espa
\defi: 
We will say that a vector space, $V$, is of
{\bf infinite dimension}~\index{dimension!infinite} if it has an infinite number 
of linearly independent vectors.

\noi 
\yaya{Examples}:

\noi
a) Let $V$ be the set of all sequences
$\{x_i\},\;i=1,\ldots,\infty$ of real numbers. This is a vector
space if we define the {\bf sum} and the {\bf product} of sequences by 
the following formula,
\beq
\{x_i\}+\alpha\{y_i\}=\{x_i+\alpha\:y_i\}.
\eeq
These vectors can also be written as
$\{x_i\}=(x_1,x_2,x_3,\ldots )$, which shows that it is the
extension of \ren with $n\to\infty$.
Clearly, the vectors $\ve{u}_1=(1,0,0,\ldots) $; $\ve{u}_2=(0,1,0,\ldots) $;
$\ve{u}_3=(0,0,1,\ldots) $; are linearly independent and infinite in number.

\espa

\noi 
b) Let $V$ be the set of continuous functions on the interval
$[0,1]$. This is a vector space since if $f$ and $g$ are
continuous on $[0,1]$, then $h=f+\alpha g$, $\alpha \in \re$, 
is also continuous on $[0,1]$.

The set of vectors $u_n=x^n\,,\;n\in{\ve N}$, is linearly
independent and infinite
$(\sum^M_{n=0}c^n\,u_n=\sum_{n=0}^Mc^n\,x^n=0\Longrightarrow
c^n=0\;\;\forall \;n\leq  M)$ since a polynomial of degree
$M$ has at most $M$ roots.

Infinite-dimensional spaces can also be assigned
norms, but in this case, these norms are not equivalent, and therefore
one must be careful not to confuse the resulting structures.
To avoid confusion, we will assign different names to spaces with different norms.

\noi\yaya{Examples}:
\noi 
a) The normed vector space $l^2$ is the space of
infinite sequences with the norm
$\|\{x_i\}\|_2=\sqrt{\sum_{i=1}^{\infty} x_i^2} < \infty $.

\noi 
b) The normed vector space $l^{\infty}$ is the space of
infinite sequences with the norm
$\|\{x_i\}\|_{\infty}=sup_{i}\{|x_i|\}$. That is, the space of all
bounded sequences.

\noi c) The normed vector space $C[a,b]$ is the space of
continuous functions with the norm $\|f\|_c=sup_{x\in [a,b]}\{|f(x)|\}$.
That is, the space of bounded continuous functions on $[a,b]$.

\espa
\noi\yaya{Exercises}:

\noi
1) Show that the norms defined in the previous example are indeed norms.

\noi
2) Show that there are sequences in $l^{\infty}$ that do not belong to
$l^2$. Hint: Find one of them.

\noi
3) Show that $\|\{x_i\}\|_n := \sqrt[n]{\sum_{i=1}^{\infty} |x_i|^n}_{\stackrel{
\textstyle \longrightarrow}{\scriptstyle n \to \infty}} sup_i \{|x_i|\}$.
\espa

Unlike the finite-dimensional case, an infinite-dimensional normed space is not necessarily complete. To illustrate this, consider the normed vector space $l_0^{\infty}$, which is a subspace of $l^{\infty}$ consisting of all bounded sequences with only a finite number of non-zero terms. Each sequence $\{x_i\}_n= (1,1/2,1/3,\ldots,1/n,0,0,\ldots) $ is in
$l_0^{\infty}$, the sequence of sequences $\{\{x_i\}_n\}$ is Cauchy 
$$\lp \|\{x_i\}_{m}-\{x_i\}_n\|_{\infty}=\frac{1}{n+1}_{\stackrel{\textstyle\longrightarrow}
{\scriptstyle n\to\infty}} 0 \rp$$ 
and converges to the sequence $\{x_i\}_{\infty}=( 1,1/2,
1/3 ,\ldots)\in l^{\infty}$ {\it which does not belong to} $l_0^{\infty}$. Therefore,
$l_0^{\infty}$ is not complete.

\espa
\defi: 
We will say that a normed vector space
$(V,\|\cdot\|)$ is a {\bf Banach space}~\index{Banach!space} if it is complete.

\noi
\yaya{Examples}: 
\ren with any of its norms, $l^2$ and
$l^{\infty}$ are Banach spaces.
\espa

An important result, crucial in the proof of the Fundamental Theorem,
is the following theorem.
\bteo
The vector space of bounded continuous functions, $C[a,b]$, is complete.
\eteo

\pru: Let $\{f_n(x)\}$ be a Cauchy sequence in $C[a,b]$,
that is, each $f_n(x)$ is a continuous and bounded function on
$[a,b]$ and it holds that given $\eps>0$ there exists $N>0$ such that
$\forall\;m,n>N\;\;\; \|f_n-f_m\|_c=sup_{x\in[a,b]}|f_n(x)-f_m(x)|<\eps$.
But then for each $x\in[a,b]$ the sequence of real numbers
$\{f_n(x)\}$ is Cauchy. But the real numbers are complete and
therefore for each $x\in[a,b]\;\;\{f_n(x)\}$ converges to a
number that we will call $f(x)$. But then given $\eps>0$ for
all $N$ such that $m,n\geq N$ implies $\|f_n-f_m\|_c<\eps$, we have
that 

\beq
    \barr{rcl}\dip
            sup_{x\in[a,b]}|f(x)-f_N(x)| & = & sup_{x\in[a,b]}\dip\lim_{n\to\infty} |f_n(x)-f_N(x)| \\
                    & \leq & sup_{n\geq N} sup_{x\in[a,b]}  |f_n(x)-f_N(x)| \\             
                    & = & sup_{n\geq N} \|f_n-f_N\|_c<\eps .
          \earr
\eeq

Therefore, if we could prove that $f\in C[a,b]$, then
we would have that $\|f-f_n\|_c\to 0,\;\;n\to\infty$ and therefore that
$\{f_n\}\to f$ in $C[a,b]$ and the theorem would be proven.

Let $x\in[a,b]$ be any point, we will prove that $f$ is continuous at
$x$ and thus on all $[a,b]$. Let $\eps>0$, we want to find
a $\del $ such that  $|x-y|<\del$ implies $|f(x)-f(y)|<\eps$.
Take $N$ such that $\|f-f_N\|_c<\eps/3$  and $\del$ such that
$|x-y|<\del$ implies $|f_N(x)-f_N(y)|<\eps/3$ [This is possible since
the $f_N(x)$ are continuous on $[a,b]$]. Then $|x-y|<\del$ implies
{\small
\beq\barr{rcl}
|f(x)-f(y)|&\leq&|f(x)-f_N(x)|+|f_N(x)-f_N(y)|+|f_N(y)-f(y)| \\
           &<   & \frac13\eps + \frac13\eps +\frac13 \eps=\eps .
\earr
\eeq
}
%
and therefore the continuity of $f(x)$. Since $[a,b]$ is a compact set
(closed and bounded), $f$ is bounded on $[a,b]$ and therefore belongs
to $C[a,b]$ 
\epru
\espa

\defi: 
Let $T:V\to V$ be a map from a Banach space $V$ to itself. We will say that $T$ is a {\bf
contraction}~\index{contraction} if there exists $\lambda<1$ such that:
\beq 
\|T(\ve{x})- T(\ve{y})\|_V\leq\lambda\,\|\ve{x}-\ve{y}\|_V.
\eeq

\noi\yaya{Examples}:

a) The linear map in $l^2$; $\ve{A}\{\ve{x}_i\}=\lb \frac{\ve{x}_i}{i+1}\rb$.

b) The map from $\re^2$ to $\re^2$ that sends the point $(x,y)$ to the
point $(x_0+x/2,y/2)$.

c) Any Lipschitz function from $\re$ to $\re$ with a modulus of
continuity ($k$) less than one.
\espa

The important property of these maps is the following theorem

\bteo 
\label{Teorema_del_mapa_contractivo}
Let $T:V\to V$ be a contraction. Then there exists a unique
$\ve{v}\in V$ such that $T(\ve{v})=\ve{v}$, and the sequence $T^n(\ve{u})$
converges to $\ve{v}$ for any $\ve{u}$.
\eteo

\pru:
Let $\|T(\ve{u})-\ve{u}\|=d$, then 
$\|T^{n+1}(\ve{u})- T^n(\ve{u})\|\leq\lambda^n\,d$ and if $m>n$

{%\small
\[
\barr{rcl}
\|&&\!\!\!\!\!\!\!\!\!\!\!\!\!\!\!\!\!\!\!\!\!T^m (\ve{u}) - T^n(\ve{u})\| \\
&=& \|T^m(\ve{u})-T^{m-1}(\ve{u})+T^{m-1}(\ve{u})-T^{m-2}(\ve{u})+\cdots T^n(\ve{u})\| \\
 & \leq &  \|T^m(\ve{u})-T^{m-1}(\ve{u})\|+\|T^{m-1}(\ve{u})-T^{m-2}(\ve{u})\|
+\cdots  \\
 & \leq &  d\;\sum^m_n \lambda^m. 
 \earr 
\]
}
%
Since $\sum_{n=0}^{\infty}\lambda^n$ converges, $\{T^n(\ve{u})\}$ is a Cauchy sequence.
But $V$ is complete and therefore there exists $\ve{v}\in V$ such
that $\lim_{n\to\infty}T^n\,\ve{u}=\ve{v}$.

Since every contraction is continuous, we have that $$T(\ve{v}) = T (
\lim_{n\to\infty} T^n(\ve{u}))=\lim_{n\to\infty}T^{n+1}(\ve{u})=\ve{v}.$$

It only remains to prove that $\ve{v}$ is unique. Suppose by contradiction
that there exists $\ve{w}\in V$ different from $\ve{v}$ and such that $T\,\ve{w}=\ve{w}$. 
But
then 
$\|\ve{w}-\ve{v}\|_V=\|T(\ve{w})-T(\ve{v})\|_V\leq\lambda\,\|\ve{w}-\ve{v}\|_V$ 
which is a contradiction since $\lambda\neq 1$ 
\epru

\ejer: 
Let $T: B_{R,x_0} \to B_{R,x_0}$, $B_{R,x_0} \in V$ be a closed ball
of radius $R$ around $x_0$, a contraction~\footnote{Also adjust the definition of a contraction for this case.}. 
Prove for
this case the same statements as in the previous theorem.
\espa


\paragraph{Proof of points $i$) and $ii)$ of the fundamental theorem.}

Since we only want to see local existence and uniqueness, that is, only in
a neighborhood of a point $p$ of $M$, it is sufficient to consider the system
in $\re^n$. This will allow us to use the Euclidean norm present there.
To see this, take a chart $(U,\varphi)$ with $p\in U$ and for
simplicity $\varphi(p)=0\,\in\re^n$. Using this map, we can
translate the vector field $\ve{v}$ in $M$ to a vector field $\ti
v$ defined in a neighborhood of zero in \ren.  There we will treat the
problem of finding its integral curves $g(t,x)$ that pass through the
point $x\in\varphi(U)$ at time $t=0$. Then, through the map
$\varphi^{-1}$, we will obtain in $M$ the one-parameter families of
diffeomorphisms  $g^t(q)$ , $q\in U$, which will be tangent at every
point to the vector field $\ve{v}$.

With this in mind, it only remains to see that for $R>0$ and $\eps>0$
sufficiently small 
there exist integral curves, 
$g(t,x):[0,\eps] \times B_R=\{x\in \ren
\;|\;\|x\|_V < R  \} \to \ren$,  
of the vector $\ti v$, that is, maps satisfying
\beq  
\derc{g(t,x)}{t}=\ti v(g(t,x))\;\;\;,\;\;\;\;\;g(0,x)=x,
\label{7*}\eeq
with $g(t,x)$ continuous with respect to the second argument, that is, with
respect to the initial condition. By the assumption that $v$
is Lipschitz~\footnote{To prove the local existence and uniqueness of solutions, it is only 
necessary to assume that $\ve{v}$ is Lipschitz.}, 
we have that there exists $k>0$ such that $\forall\;x,y\in B_R$
\beq
\|\ti v(x)-\ti v(y)\|_V<k\,\|x-y\|_V\;.\eeq

Now consider the Banach space $C([0,\eps]\times B_R)$, which consists
of all continuous maps (in $t$ and $x$) from $[0,\eps]\times B_R$
to \ren{} with the norm 
\beq\barr{rccl}
     \|h\|_C & = & sup & \|h(t,x)\|_V\;\;, \\
             &   & x\in B_R & \\
             &   & t\in[0,\eps].
             \earr
\eeq

Let the map of the ball of radius $R$ in $C([0,\eps ]\times B_R)$ into itself be given by,
\beq
T(h)=\int_0^t \ti v(x+h(\tau,x))\,d\tau\;\;.
\eeq
For this map to be well-defined, we will assume $R$ is
sufficiently small so that $\ti v$ is defined and
satisfies the Lipschitz condition in $B_{2R}$ and $\eps$ is less than
$R/C$, where \hfill 
$C=\max_{x\in B_{2R}} |\ti v(x)|$, so that 
if $\|h\|_C < R$ then $\|x+h(\tau,x)\|_V < 2R \;\;\forall \tau\in[0,\eps]$
and therefore
$\|T(h)\|_C < R$ in all $C([0,\eps ]\times B_R)$. [See figure 7.1.] 

\espa 
%\fig{6cm}{Entornos usados en la prueba del Teorema Fundamental.}

\begin{figure}[htbp]
  \begin{center}
    \resizebox{7cm}{!}{\myinput{Figure/m7_1}}
    \caption{Neighborhoods used in the proof of the Fundamental Theorem.}
    \label{fig:7_1}
   \end{center}
\end{figure}

We will only prove points $i)$ and $ii)$, for which we will use the
method of successive approximations by Picard, which is important in applications. The proof of point $iii)$ is of a technical nature and does not provide any additional significant insight. Point $iv)$ follows from points $i)$ and $ii)$, and its proof is identical to that of Corollary 1.2 of the analogous Theorem for the case of an ODE.

To prove these points, we will need to develop some ideas and results from the mathematical theory of infinite-dimensional vector spaces. We will restrict ourselves to the minimum required by the theorem since these topics will be developed more extensively in the second part of this course.

\espa
\defi: 
We will say that a vector space, $V$, is of
{\bf infinite dimension}~\index{dimension!infinite} if it has an infinite number 
of linearly independent vectors.

\noi 
\yaya{Examples}:

\noi
a) Let $V$ be the set of all sequences
$\{x_i\},\;i=1,\ldots,\infty$ of real numbers. This is a vector
space if we define the {\bf sum} and the {\bf product} of sequences by 
the following formula,
\beq
\{x_i\}+\alpha\{y_i\}=\{x_i+\alpha\:y_i\}.
\eeq
These vectors can also be written as
$\{x_i\}=(x_1,x_2,x_3,\ldots )$, which shows that it is the
extension of \ren with $n\to\infty$.
Clearly, the vectors $\ve{u}_1=(1,0,0,\ldots) $; $\ve{u}_2=(0,1,0,\ldots) $;
$\ve{u}_3=(0,0,1,\ldots) $; are linearly independent and infinite in number.

\espa

\noi 
b) Let $V$ be the set of continuous functions on the interval
$[0,1]$. This is a vector space since if $f$ and $g$ are
continuous on $[0,1]$, then $h=f+\alpha g$, $\alpha \in \re$, 
is also continuous on $[0,1]$.

The set of vectors $u_n=x^n\,,\;n\in{\ve N}$, is linearly
independent and infinite
$(\sum^M_{n=0}c^n\,u_n=\sum_{n=0}^Mc^n\,x^n=0\Longrightarrow
c^n=0\;\;\forall \;n\leq  M)$ since a polynomial of degree
$M$ has at most $M$ roots.

Infinite-dimensional spaces can also be assigned
norms, but in this case, these norms are not equivalent, and therefore
one must be careful not to confuse the resulting structures.
To avoid confusion, we will assign different names to spaces with different norms.

\noi\yaya{Examples}:
\noi 
a) The normed vector space $l^2$ is the space of
infinite sequences with the norm
$\|\{x_i\}\|_2=\sqrt{\sum_{i=1}^{\infty} x_i^2} < \infty $.

\noi 
b) The normed vector space $l^{\infty}$ is the space of
infinite sequences with the norm
$\|\{x_i\}\|_{\infty}=sup_{i}\{|x_i|\}$. That is, the space of all
bounded sequences.

\noi c) The normed vector space $C[a,b]$ is the space of
continuous functions with the norm $\|f\|_c=sup_{x\in [a,b]}\{|f(x)|\}$.
That is, the space of bounded continuous functions on $[a,b]$.

\espa
\noi\yaya{Exercises}:

\noi
1) Show that the norms defined in the previous example are indeed norms.

\noi
2) Show that there are sequences in $l^{\infty}$ that do not belong to
$l^2$. Hint: Find one of them.

\noi
3) Show that $\|\{x_i\}\|_n := \sqrt[n]{\sum_{i=1}^{\infty} |x_i|^n}_{\stackrel{
\textstyle \longrightarrow}{\scriptstyle n \to \infty}} sup_i \{|x_i|\}$.
\espa

Unlike the finite-dimensional case, an infinite-dimensional normed space is not necessarily complete. To illustrate this, consider the normed vector space $l_0^{\infty}$, which is a subspace of $l^{\infty}$ consisting of all bounded sequences with only a finite number of non-zero terms. Each sequence $\{x_i\}_n= (1,1/2,1/3,\ldots,1/n,0,0,\ldots) $ is in
$l_0^{\infty}$, the sequence of sequences $\{\{x_i\}_n\}$ is Cauchy 
$$\lp \|\{x_i\}_{m}-\{x_i\}_n\|_{\infty}=\frac{1}{n+1}_{\stackrel{\textstyle\longrightarrow}
{\scriptstyle n\to\infty}} 0 \rp$$ 
and converges to the sequence $\{x_i\}_{\infty}=( 1,1/2,
1/3 ,\ldots)\in l^{\infty}$ {\it which does not belong to} $l_0^{\infty}$. Therefore,
$l_0^{\infty}$ is not complete.

\espa
\defi: 
We will say that a normed vector space
$(V,\|\cdot\|)$ is a {\bf Banach space}~\index{Banach!space} if it is complete.

\noi
\yaya{Examples}: 
\ren with any of its norms, $l^2$ and
$l^{\infty}$ are Banach spaces.
\espa

An important result, crucial in the proof of the Fundamental Theorem,
is the following theorem.
\bteo
The vector space of bounded continuous functions, $C[a,b]$, is complete.
\eteo

\pru: Let $\{f_n(x)\}$ be a Cauchy sequence in $C[a,b]$,
that is, each $f_n(x)$ is a continuous and bounded function on
$[a,b]$ and it holds that given $\eps>0$ there exists $N>0$ such that
$\forall\;m,n>N\;\;\; \|f_n-f_m\|_c=sup_{x\in[a,b]}|f_n(x)-f_m(x)|<\eps$.
But then for each $x\in[a,b]$ the sequence of real numbers
$\{f_n(x)\}$ is Cauchy. But the real numbers are complete and
therefore for each $x\in[a,b]\;\;\{f_n(x)\}$ converges to a
number that we will call $f(x)$. But then given $\eps>0$ for
all $N$ such that $m,n\geq N$ implies $\|f_n-f_m\|_c<\eps$, we have
that 

\beq
    \barr{rcl}\dip
            sup_{x\in[a,b]}|f(x)-f_N(x)| & = & sup_{x\in[a,b]}\dip\lim_{n\to\infty} |f_n(x)-f_N(x)| \\
                    & \leq & sup_{n\geq N} sup_{x\in[a,b]}  |f_n(x)-f_N(x)| \\             
                    & = & sup_{n\geq N} \|f_n-f_N\|_c<\eps .
          \earr
\eeq

Therefore, if we could prove that $f\in C[a,b]$, then
we would have that $\|f-f_n\|_c\to 0,\;\;n\to\infty$ and therefore that
$\{f_n\}\to f$ in $C[a,b]$ and the theorem would be proven.

Let $x\in[a,b]$ be any point, we will prove that $f$ is continuous at
$x$ and thus on all $[a,b]$. Let $\eps>0$, we want to find
a $\del $ such that  $|x-y|<\del$ implies $|f(x)-f(y)|<\eps$.
Take $N$ such that $\|f-f_N\|_c<\eps/3$  and $\del$ such that
$|x-y|<\del$ implies $|f_N(x)-f_N(y)|<\eps/3$ [This is possible since
the $f_N(x)$ are continuous on $[a,b]$]. Then $|x-y|<\del$ implies
{\small
\beq\barr{rcl}
|f(x)-f(y)|&\leq&|f(x)-f_N(x)|+|f_N(x)-f_N(y)|+|f_N(y)-f(y)| \\
           &<   & \frac13\eps + \frac13\eps +\frac13 \eps=\eps .
\earr
\eeq
}
%
and therefore the continuity of $f(x)$. Since $[a,b]$ is a compact set
(closed and bounded), $f$ is bounded on $[a,b]$ and therefore belongs
to $C[a,b]$ 
\epru
\espa

\defi: 
Let $T:V\to V$ be a map from a Banach space $V$ to itself. We will say that $T$ is a {\bf
contraction}~\index{contraction} if there exists $\lambda<1$ such that:
\beq 
\|T(\ve{x})- T(\ve{y})\|_V\leq\lambda\,\|\ve{x}-\ve{y}\|_V.
\eeq

\noi\yaya{Examples}:

a) The linear map in $l^2$; $\ve{A}\{\ve{x}_i\}=\lb \frac{\ve{x}_i}{i+1}\rb$.

b) The map from $\re^2$ to $\re^2$ that sends the point $(x,y)$ to the
point $(x_0+x/2,y/2)$.

c) Any Lipschitz function from $\re$ to $\re$ with a modulus of
continuity ($k$) less than one.
\espa

The important property of these maps is the following theorem

\bteo 
\label{Teorema_del_mapa_contractivo}
Let $T:V\to V$ be a contraction. Then there exists a unique
$\ve{v}\in V$ such that $T(\ve{v})=\ve{v}$, and the sequence $T^n(\ve{u})$
converges to $\ve{v}$ for any $\ve{u}$.
\eteo

\pru:
Let $\|T(\ve{u})-\ve{u}\|=d$, then 
$\|T^{n+1}(\ve{u})- T^n(\ve{u})\|\leq\lambda^n\,d$ and if $m>n$

{%\small
\[
\barr{rcl}
\|&&\!\!\!\!\!\!\!\!\!\!\!\!\!\!\!\!\!\!\!\!\!T^m (\ve{u}) - T^n(\ve{u})\| \\
&=& \|T^m(\ve{u})-T^{m-1}(\ve{u})+T^{m-1}(\ve{u})-T^{m-2}(\ve{u})+\cdots T^n(\ve{u})\| \\
 & \leq &  \|T^m(\ve{u})-T^{m-1}(\ve{u})\|+\|T^{m-1}(\ve{u})-T^{m-2}(\ve{u})\|
+\cdots  \\
 & \leq &  d\;\sum^m_n \lambda^m. 
 \earr 
\]
}
%
Since $\sum_{n=0}^{\infty}\lambda^n$ converges, $\{T^n(\ve{u})\}$ is a Cauchy sequence.
But $V$ is complete and therefore there exists $\ve{v}\in V$ such
that $\lim_{n\to\infty}T^n\,\ve{u}=\ve{v}$.

Since every contraction is continuous, we have that $$T(\ve{v}) = T (
\lim_{n\to\infty} T^n(\ve{u}))=\lim_{n\to\infty}T^{n+1}(\ve{u})=\ve{v}.$$

It only remains to prove that $\ve{v}$ is unique. Suppose by contradiction
that there exists $\ve{w}\in V$ different from $\ve{v}$ and such that $T\,\ve{w}=\ve{w}$. 
But
then 
$\|\ve{w}-\ve{v}\|_V=\|T(\ve{w})-T(\ve{v})\|_V\leq\lambda\,\|\ve{w}-\ve{v}\|_V$ 
which is a contradiction since $\lambda\neq 1$ 
\epru

\ejer: 
Let $T: B_{R,x_0} \to B_{R,x_0}$, $B_{R,x_0} \in V$ be a closed ball
of radius $R$ around $x_0$, a contraction~\footnote{Also adjust the definition of a contraction for this case.}. 
Prove for
this case the same statements as in the previous theorem.
\espa


\paragraph{Proof of points $i$) and $ii)$ of the fundamental theorem.}

Since we only want to see local existence and uniqueness, that is, only in
a neighborhood of a point $p$ of $M$, it is sufficient to consider the system
in $\re^n$. This will allow us to use the Euclidean norm present there.
To see this, take a chart $(U,\varphi)$ with $p\in U$ and for
simplicity $\varphi(p)=0\,\in\re^n$. Using this map, we can
translate the vector field $\ve{v}$ in $M$ to a vector field $\ti
v$ defined in a neighborhood of zero in \ren.  There we will treat the
problem of finding its integral curves $g(t,x)$ that pass through the
point $x\in\varphi(U)$ at time $t=0$. Then, through the map
$\varphi^{-1}$, we will obtain in $M$ the one-parameter families of
diffeomorphisms  $g^t(q)$ , $q\in U$, which will be tangent at every
point to the vector field $\ve{v}$.

With this in mind, it only remains to see that for $R>0$ and $\eps>0$
sufficiently small 
there exist integral curves, 
$g(t,x):[0,\eps] \times B_R=\{x\in \ren
\;|\;\|x\|_V < R  \} \to \ren$,  
of the vector $\ti v$, that is, maps satisfying
\beq  
\derc{g(t,x)}{t}=\ti v(g(t,x))\;\;\;,\;\;\;\;\;g(0,x)=x,
\label{7*}\eeq
with $g(t,x)$ continuous with respect to the second argument, that is, with
respect to the initial condition. By the assumption that $v$
is Lipschitz~\footnote{To prove the local existence and uniqueness of solutions, it is only 
necessary to assume that $\ve{v}$ is Lipschitz.}, 
we have that there exists $k>0$ such that $\forall\;x,y\in B_R$
\beq
\|\ti v(x)-\ti v(y)\|_V<k\,\|x-y\|_V\;.\eeq

Now consider the Banach space $C([0,\eps]\times B_R)$, which consists
of all continuous maps (in $t$ and $x$) from $[0,\eps]\times B_R$
to \ren{} with the norm 
\beq\barr{rccl}
     \|h\|_C & = & sup & \|h(t,x)\|_V\;\;, \\
             &   & x\in B_R & \\
             &   & t\in[0,\eps].
             \earr
\eeq

Let the map of the ball of radius $R$ in $C([0,\eps ]\times B_R)$ into itself be given by,
\beq
T(h)=\int_0^t \ti v(x+h(\tau,x))\,d\tau\;\;.
\eeq
For this map to be well-defined, we will assume $R$ is
sufficiently small so that $\ti v$ is defined and
satisfies the Lipschitz condition in $B_{2R}$ and $\eps$ is less than
$R/C$, where \hfill 
$C=\max_{x\in B_{2R}} |\ti v(x)|$, so that 
if $\|h\|_C < R$ then $\|x+h(\tau,x)\|_V < 2R \;\;\forall \tau\in[0,\eps]$
and therefore
$\|T(h)\|_C < R$ in all $C([0,\eps ]\times B_R)$. [See figure 7.1.] 

\espa 
%\fig{6cm}{Entornos usados en la prueba del Teorema Fundamental.}
\espa 
%\fig{6cm}{Environments used in the proof of the Fundamental Theorem.}

\begin{figure}[htbp]
  \begin{center}
    \resizebox{7cm}{!}{\myinput{Figure/m7_1}}
    \caption{Environments used in the proof of the Fundamental Theorem.}
    \label{fig:7_1}
  \end{center}
\end{figure}


\blem
If \eps{} is sufficiently small then $T$ is a contraction.
\elem

\espa
\pru:
\begin{eqnarray*}
\|T(h_1)-T(h_2)\|_V  \!\! &=& \!\!
           \int_0^t\|\ti v(x+h_1(\tau,x))-\ti v(x+h_2(\tau,x))\|_V\;d\tau \\
\!\! &\leq& \!\!\int_0^t k\,\|h_1(\tau,x)-h_2(\tau,x)\|_V \,d\tau\\
\!\! &\leq& \!\! k\,\eps\,\|h_1-h_2\|_C
\end{eqnarray*}
%                                         
and therefore
\[
\|T(h_1)-T(h_2)\|_C\leq k\,\eps\,\|h_1-h_2\|_C\;\;\;\;\;\;\;\forall
\;h_1,h_2\in C([0,\eps]\times B_R).
\]
%
Taking $\eps<1/k $ we complete the proof of the Lemma.

This Lemma and Theorem \ref{Teorema_del_mapa_contractivo} ensure 
that there exists a unique map $h(t,x)$ --the fixed point of $T$-- satisfying
\beq
h(t,x)=T(h(t,x))=\int_0^t\ti v(x+h(\tau,x))\,d\tau .
\label{7**}
\eeq

\espa
Let $g(t,x)\equiv x+h(t,x)$, this function is continuous in both
arguments --since $h(t,x)\in C([0,\eps]\times B_R)$-- and by
construction continuously
differentiable in $t$ --since it satisfies \ron{7**}--.
Differentiating \ron{7**} with respect to $t$ we see that $g(t,x)$
satisfies the equation \ron{7*} and its initial condition, which
completes the proof of points $i)$ and $ii)$ of the Fundamental Theorem.
\epru

%%%%%%%%%%%%%%%%%%%%%%%%%%%%%%%%%%%%%%%%%%%%%%%%%%%%%%%%%%%%%%%%%%%%%%%%%%%%%

%\newpage
\section{Problems}

%%%%%%%%%%%%%%%%%%%%%%%%%%%%%%%%%%%%%%%%%%%%%%%%%%%%%%%%%%%%%%%%%%%%%%%%%%%%%


%%%%%%%%%%%%%%%%%%%%%
% 1
\bpro
See that $l^2$ the unit ball is not compact. 
Hint: find an infinite sequence in the unit ball that has no accumulation point.
\epro

%2
\bpro
Prove that the condition 
\begin{equation}
  \|A(\ve{x}) - A(\ve{y})\| < \|\ve{x} - \ve{y}\|
\end{equation}
is not sufficient to guarantee the existence of a fixed point.
Hint: construct a counterexample. 
There are some very simple ones using functions from the line to itself.
\epro

%3
\bpro 
Let $f:[a,b] \to [a,b]$ be a Lipschitz function with a constant less
than one throughout the interval $[a,b]$. Prove using the fixed point theorem
for contractions that the equation $f(x) = x$ always has a
solution. Plot in a diagram $(y=f(x), x)$ the sequence given by
$x_i = f(x_{i-1})$ for the case of $f$ with positive slope and less than
one at every point. What happens when the slope becomes greater than one
in some interval?
\epro

%4
\bpro
Let $g:[a,b] \to \re$ be a continuously differentiable function such that
$g(a) < 0$, $g(b) > 0$ and $0 < c_1 < g'(x)$. Use the fixed point theorem 
for contractions to prove that there is a unique root $g(x) = 0$ in 
the interval. Hint: define the function $f(x)= x - \lambda g(x)$ for a 
conveniently chosen constant $\lambda$ and find a fixed point: 
$f(x)=x$. Note that the approximating sequence is that of the Newton method
for finding roots.
\epro


%%% Local Variables: 
%%% mode: latex
%%% TeX-master: "apu_tot.tex~/Metodos/"
%%% End: 