\chapter{Linear Systems}
\label{Linear_Systems}
\section{Homogeneous Linear System}
\label{Homogeneous_Linear_System}

\bteo
Let $A(t)$ be continuous on $I \subset \mathbb{R}$, closed. Then there exists a unique solution $\ve{x}(t): I \to V$ of the equation,
\beq
\derc{\ve{x}(t)}{t} = A(t)\,\ve{x}(t), 
\label{eqn:5.1}
\eeq
\noi with initial condition $\ve{x}(t_0)=\ve{x}_0 \in V, \; t_0 \in I$.
\eteo

\espa
\pru: 
Let $\ve{x}, \ve{y} \in \ve{V}$, then
\beq
\| \ve{A}(t)\ve{x}-\ve{A}(t)\ve{y}\|_V \leq \|\ve{A}(t)\|_{\cal L} \|\ve{x}-\ve{y}\|_V
\eeq
\noi 
and therefore $\ve{A}(t)\ve{x}$ is Lipschitz~\index{Lipschitz} in $\ve{V}$. 
The fundamental theorem guarantees local existence and uniqueness. 
The theorem will be proven if we show that $\ve{x}(t)$ cannot become infinite in finite time. For this, we use,
\beq\barr{rcl}
\dip\der{t}\|\ve{x}\|_V &=& \dip\lim_{t_1\to t}\dip\frac{\|\ve{x}(t_1)\|_V-\|\ve{x}(t)\|_V}{t_1-t}
 \leq \dip\lim_{t_1\to t}\dip\frac{\|\ve{x}(t_1)-\ve{x}(t)\|}{t_1-t} \\
        &=& \left\|\derc{\ve{x}}{t}\right\|_V = \|\ve{A}\,\ve{x}\|_V \leq \|\ve{A}\|_{\cal L} \|\ve{x}\|_V.
\earr
\eeq
\noi Integrating this inequality between $t_0$ and $t \in I$ we obtain,

\begin{equation}
\|\ve{x}(t)\|_V \leq \|\ve{x}(t_0)\|_V e^{\int_{t_0}^{t} {\|\ve{A}(\tilde t)\|_{\cal L}} d\tilde t},
\end{equation}

\noi which shows that $\ve{x}(t)$ cannot escape to infinity in finite time and therefore completes the proof.

\espa
Consider the set of solutions of \r{eqn:5.1} defined on a single closed interval $I \subset \mathbb{R}$, $ Sol(\ve{A},I)$. 
This set has the structure of a vector space. 
Indeed, if $\ve{x}(t)$ and $\ve{y}(t)$ are two solutions of \r{eqn:5.1}, then $\ve{x}(t)+\alpha \ve{y}(t), \; \alpha \in \mathbb{R}$, is also a solution. What is its dimension? The following theorem answers this question and shows that any solution of \r{eqn:5.1} can be expressed as a linear combination of a finite number $(n)=\dim \ve{V}$ of solutions.

\espa
\bteo 
$\dim Sol(\ve{A},I) = \dim \ve{V}.$
\eteo
\espa
\pru: 
Let $\{\ve{u}^0_i\}, \; i=1,\ldots,n$ be a basis of $\ve{V}$,
$t_0 \in I$ and $\{\ve{u}_i(t)\}, \; i=1,\ldots,n \; t \in I$ the set of solutions of \r{eqn:5.1} with initial condition $\ve{u}_i(t_0)=\ve{u}_i^0.$
We will show that $\{\ve{u}_i(t)\}$ form a basis of $Sol(\ve{A},I)$, and therefore 
$\dim Sol(\ve{A},I) = \dim \ve{V}$. 
First, let's see that the solutions $\{\ve{u}_i(t)\}$ are linearly independent.

Suppose there are constants $c^i$ such that 
$\ve{x}(t) = \sum_{i=0}^{n}c^i \ve{u}_i(t)$ 
is zero for some $\tilde t$ in $I$. 
Since $\ve{x}(t)$ is a solution of \r{eqn:5.1} and these are unique when an initial condition is specified, taking in this case $\ve{x}(\tilde t)=0$, we see that $\ve{x}(t)=0 \; \forall \; t \in I$. 
In particular, we have that $\ve{x}(t_0) = \sum_{i=0}^{n}c^i \ve{u}_i^0=0$ and the independence of the set $\{\ve{u}_i^0(t)\}$ implies that $c^i=0 \; \forall \; i=1,\ldots,n$, proving linear independence. 
Note that we have not only proven that $\{\ve{u}_i(t)\}$ are linearly independent as elements of $Sol(\ve{A},I)$ -- for which we would only have had to prove that if $\sum_{i=0}^{n} c^i \ve{u}_i(t)=0 \; \forall \; t \in I$ then $c^i=0 \; \forall \; i=1,\ldots,n$ which is trivial since $t_0 \in I$ -- but also that for each $t \in I$ the $\{\ve{u}_i(t)\}$ are linearly independent as elements of $\ve{V}$, this result will be used later.

To complete the theorem, let's now see that any solution of \r{eqn:5.1} $\ve{x}(t): I \to \ve{V}$, that is, any element of $Sol(\ve{A},I)$ can be written as a linear combination of $\{\ve{u}_i(t)\}$. Let $\ve{x}(t) \in Sol(\ve{A},I)$, since $\{\ve{u}^0_i\}$ form a basis of $\ve{V}$ there will be constants $c^i, \; i=1,\ldots,n$ such that $\ve{x}(t_0)=c^i \ve{u}^0_i$. Consider then the solution of \r{eqn:5.1} given by $\fit=c^i \ve{u}_i(t)$. Since $\fito=\ve{x}(t_0)$ and the solutions are unique, we have,
\beq
\ve{x}(t)=\fit=c^i \ve{u}_i(t) \; \forall \; t \in I
\eeq
\espa

The previous theorem tells us that if we know $n$ linearly independent solutions of $\ve{A}$, $\{\ve{u}_i(t)\}$, we know all its solutions, since these will be linear combinations of $\{\ve{u}_i(t)\}$. The dependence on the initial data is also linear, that is, if $\ve{x}(t)$, $\ve{y}(t) \in Sol(\ve{A},I)$ and $\ve{x}(t_0)=\ve{x}_0$ and $\ve{y}(t_0)=\ve{y}_0$ then $\ve{x}(t)+\alpha \ve{y}(t), \; \alpha \in \mathbb{R}$, is the solution with initial data $\ve{x}_0+\alpha \ve{y}_0$ and therefore the map $g^t_{t_0}$ -- which in this case we will call $\ve{X}_{t_0}^t$ -- that takes initial data given at $t=t_0$ to solutions with those data, at time $t$ is a linear operator from $\ve{V}$ to $\ve{V}$. 
Indeed, if $\{\ve{\theta}_0^i\}$ is the co-basis of the basis $\{\ve{u}_i\}$, i.e. $\ve{\theta}_0^j(\ve{u}_i^0)=\delta^j_i$. 
Then $\ve{X}_{t_0}^t = \sum_{i=1}^n \ve{u}_i(t) \ve{\theta}^i_0$. 
Due to the linear independence of $\{\ve{u}_i(t)\}$ as elements of $\ve{V}$, it can be seen that the map $\ve{X}_{t_0}^t: \ve{V} \to \ve{V}$ is injective and therefore invertible, for each $t \in I$. Its inverse, which we will denote $(\ve{X}_{t_0}^t)^{-1}$, takes the solution at $t$ to its initial data at $t=t_0$, that is, $(\ve{X}_{t_0}^t)^{-1}=\ve{X}_t^{t_0}$.

\espa
\ejer:
Prove that $(\ve{X}_{t_0}^t)^{-1}=\ve{X}_t^{t_0}$.

\ejer:
Prove that $\ve{X}_{t_0}^t$ does not depend on the basis used.
\espa

The map $\ve{X}_{t_0}^t$ is actually the one-parameter family of diffeomorphisms from $\ve{V}$ to $\ve{V}$ generated by the vector field $\ve{A}\ve{x}$. 
In this case, these are linear. 
Indeed, $\ve{X}_{t_0}^t \ve{x}_0$ is the curve that passes through the point $\ve{x}_0 \in \ve{V}$ at $t=t_0$ and whose tangent vector is $\ve{A} \ve{X}_{t_0}^t \ve{x}_0$.

We have seen then that if we know a set of $n$ linearly independent solutions $\{\ve{u}_i(t)\}$ we can construct any solution using the operator $\ve{X}_{t_0}^t$ applied to the initial data of our choice.

How do we know in practice that a set of $n$ solutions $\{\ve{u}_i(t)\}$ is linearly independent? As we have seen, it is sufficient to see that these are linearly independent, as elements of $\ve{V}$, at any time $t$. That is, the scalar
\beq
w(t)=\varepsilon(\ve{u}_1(t),\ve{u}_2(t),\ldots,\ve{u}_n(t))
\eeq
\noi is different from zero. This function is called the {\bf Wronskian}~\index{Wronskian} of the system and satisfies a particularly simple equation whose solution, called the {\bf Liouville formula},~\index{Liouville formula} is 
\beq
w(t)=w(t_0) e^{\dip\int_{t_0}^t tr(\ve{A}(\tilde t)) d\tilde t}
\label{5.2}
\eeq
\espa

\pru:
\beq
\barr{rcl}
\!\!\!\!\! \dot w(t) & = & \varepsilon(\dot{\ve{u}}_1,\ve{u}_2,\ldots,\ve{u}_n) 
            +  \varepsilon(\ve{u}_1,\dot{\ve{u}}_2,\ldots,\ve{u}_n)    
           +\cdots + 
          \varepsilon(\ve{u}_1,\ve{u}_2,\ldots,\dot{\ve{u}}_n) \\
 & = & \varepsilon(\ve{A}\ve{u}_1,\ve{u}_2,\ldots,\ve{u}_n)  +  \varepsilon(\ve{u}_1,\ve{A}\ve{u}_2,\ldots,\ve{u}_n) + \cdots + \varepsilon(\ve{u}_1,\ve{u}_2,\ldots,\ve{A}\ve{u}_n) \\
& = & tr(\ve{A}) w(t), 
\earr
\eeq                                            

\noi whose solution is \r{5.2}.

\espa

The Liouville formula is an independent demonstration of the result that $\{\ve{u}_i(t)\}$ form a basis of $\ve{V}$ for each $t \in I$. Note that if $tr(\ve{A}) \equiv 0$ the Wronskian is constant and can be useful to determine a solution in terms of others already known.

%%%%%%%%%%%%%%%%%%%%%%%%%%%%%%%%%%%%%%%%%%%%%%%%%%%%%%%%%%%%%%%%%%%%%%%%%%%%

\section{Inhomogeneous Linear System -- Variation of Constants}

%%%%%%%%%%%%%%%%%%%%%%%%%%%%%%%%%%%%%%%%%%%%%%%%%%%%%%%%%%%%%%%%%%%%%%%%%%%%

Here we will deal with the system,
\beq
\derc{\ve{x}}{t}= \ve{A}(t)\,\ve{x}+\ve{b}(t),
\label{5.3}
\eeq
\noi where $\ve{b}(t):I\to \ve{V}$ is integrable. We will see that if we know the
operator $\ve{X}_{t_0}^t$ corresponding to the homogeneous system
\beq
\derc{\ve{x}}{t}= \ve{A}(t)\,\ve{x},
\label{5.4}
\eeq
\noi Then we know all the solutions of system \r{5.3}. The
method we will use is called 
{\bf variation of constants}~\index{variation of constants} and
it is also useful for obtaining approximate solutions to
non-linear systems. The method consists of assuming that the
solution of \r{5.3} will be of the form,
\beq
\fip(t)=\xto\,\ve{c}(t)
\eeq
\noi for some map $\ve{c}(t):I\to \ve{V}$, differentiable. 
Note that if
$\ve{c}(t)=c\in \ve{V}$, that is, a constant map, 
then $\fip(t)$ satisfies \r{5.4}.
Substituting $\fip(t)$ in \r{5.3} we get,
\beq
\barr{rclcl}
 \der{t}\dot{\fip}(t) & = & \left(\der{t}\xto\right)\,\ve{c}(t) +
                                           \xto\der{t}\ve{c}(t) & & \\
      & = & \ve{A}(t)\xto\,\ve{c}(t) + \xto\der{t}\ve{c}(t) &= & \ve{A}(t)\xto\,\ve{c}(t)
                                                       +\ve{b}(t). 
\earr
\eeq
\noi From which we see that for $\fit$ to be a solution, $\ve{c}(t)$
must satisfy the equation $\xto\der{t}\ve{c}(t)=\ve{b}(t).$

But since $\xto$ is invertible and its inverse is $\xot$ we get
\beq
\der{t} \ve{c}(t) =\xot\,\ve{b}(t).
\eeq

\noi Integrating we get
\beq
\ve{c}(t)=\int_{t_0}^t \xoti \ve{b}(\tilde t) \;d\tilde t + c(t_0)
\eeq
\noi or
\beq
\fit=\xto\left[\int_{t_0}^t\;\xoti\,\ve{b}(\tilde t)\;d\tilde t\right] +
\xto\,\fito ,
\label{5.6}
\eeq
\noi where $\fito$ is any initial condition.

\espa

%***************************************************************************
Due to the existence theorem of solutions of the homogeneous system
we know that $\xot$ exists and is differentiable $\forall\;t\in I$ and therefore
$\fit$ also exists and is differentiable $\forall\; t\in I$,
since $\ve{b}(\tilde t)$ is integrable. Since the initial data for $\fit$
can be given arbitrarily, and the solutions of \r{5.3} are
unique, we conclude that \r{5.6} is the general solution of system \r{5.3}.




\section{Homogeneous Linear Systems: Constant Coefficients}
\label{Homogeneous_Linear_Systems:_Constant_Coefficients}

The equation we will deal with here is 
\beq
\derc{\ve{x}}{t}=\ve{A}\,\ve{x},      \label{5.7}
\eeq

\noi where $\ve{x} :I\su \re\to \ve{V}$ is a curve in $\ve{V}$ and $\ve{A}$ is a
linear operator of $\ve{V}$, $\ve{A}:\ve{V}\to \ve{V}$.

We already saw in an exercise that $\ve{x}(t)=e^{\ve{A}t}\,\ve{x}_0$, $\ve{x}_0\in \ve{V}$
is a solution of~\ref{5.7} with initial condition
$\ve{x}(0)=\ve{x}_0$. Note that since $e^{\ve{A}t}$ is defined for all
$t\in\re$, the solutions we have found are also
defined in all $\re$.

Let $\tilde{\ve{x}}(t)$ be any solution of \r{5.7} defined in an
interval $\ti I\su\re$ and let $t_1\in\ti I$. Then the curve
$\ve{x}(t)=e^{\ve{A}(t_1-t)}\,\ti{\ve{x}} (t_1)$ is also a solution
of \r{5.7} and $\ve{x}(t_1)=\ti{\ve{x}}(t_1)$, but by the uniqueness of the solutions
of \r{5.7}, $\ve{x}(t)$ and $\ti{\ve{x}}(t)$ must coincide in all $\ti I$ and
$\ve{x}(t)$ is the maximum extension of $\ti{\ve{x}}(t)$. Thus we see that
through the exponential $e^{\ve{A}t}$ we obtain all the solutions
of \r{5.7}. In fact, we have shown that the one-parameter family
of linear diffeomorphisms $g^t=e^{\ve{A}t}$ is globally defined
[$\forall t\in\re,\forall \ve{x}_0\in \ve{V}$] and is the generator of the vector
field $v(\ve{x})=\ve{A} \ve{x}$. 
[With the definition given in \ref{Homogeneous_Linear_System} in this case
$\xto=g^{t-t_0}=e^{\ve{A}(t-t_0)}$.]
\espa

%%%%%%%%%%%%%%%%%%%%%%%%%%%%%%%%%%%%%%%%%%%%%%%%%%%%%%%%%%%%%%%%%%%%%%%%%

\noi What is the general form of this family? Or in other words,
what is the functional form of the exponential map?
For this it is convenient to take a fixed basis $\{\ve{u}_i\}$
(independent of $t$) in $V$ 
and express the equation in terms of the n-tuple of components $\{x^i\}$
of $\ve{x}$.
In this way we obtain,
\beq
\derc{x^i}{t}= \sum_{j=1}^n A^i{}_j \, x^j,      \label{eqn:5.7_b}
\eeq
that is, an equation for the n-tuple $\{x^i\}$.
The advantage of this change is that we can take a convenient basis,
in particular the orthogonal basis of Lemma \ref{2_lem_Schur} (of Schur)
in which the matrix $A^i{}_j$ is upper triangular.
The price paid for the simplification is that, since the Schur basis is generally complex, both the n-tuple $\{x^i\}$ and the
components of $\ve{A}$, $A^i{}_j$, are complex and therefore
we have moved to a problem in $\Complex^n$, or abstractly
in the complexification of $V$.

For that basis we have that the nth component satisfies a
particularly simple equation!,
\beq
\derc{x^n}{t}=  A^n{}_n \, x^n    
\label{eqn:5.7_b_n}
\eeq
and therefore, $x^n(t) =  e^{\lambda_n t}x^n_0$ (remember that the 
diagonal of an upper triangular matrix is composed of its eigenvalues),
where $x^n_0$ is the nth component of the initial condition at $t=0$.
The equation for the (n-1)th component is,


\begin{eqnarray}
\derc{x^{n-1}}{t} &=&  A^{n-1}{}_{n-1} \, x^{n-1} + A^{n-1}{}_n \, x^n \\
                  &=& \lambda_{n-1} \, x^{n-1} + A^{n-1}{}_n \,e^{\lambda_n t}x^n_0 .
\label{eqn:5.7_b_n-1}
\end{eqnarray}
Using the formula (\ref{5.6}) for the one-dimensional case, (and $\ve{A}$
constant) we get,

\begin{eqnarray}
  x^{n-1}(t) &=& e^{\lambda_{n-1}t} x^{n-1}_0  
               + e^{\lambda_{n-1}t} 
               \int_0^t e^{-\lambda_{n-1}s}(A^{n-1}{}_n \,e^{\lambda_n s} x^n_0 )\; ds \\
               &=& e^{\lambda_{n-1}t} x^{n-1}_0  
               + e^{\lambda_{n-1}t} 
               \int_0^t e^{(\lambda_n -\lambda_{n-1})s}\; ds \;\;A^{n-1}{}_n \, x^n_0,
\end{eqnarray}
and we see that if $\lambda_n -\lambda_{n-1} \neq 0$ then the solution
is a sum of exponentials, 
$x^{n-1}(t) = e^{\lambda_{n-1}t} x^{n-1}_0 + \frac{A^{n-1}{}_n \, x^n_0}{\lambda_n -\lambda_{n-1}}(e^{\lambda_{n}t}-e^{\lambda_{n-1}t})$.
If the eigenvalues coincide ($\lambda_n -\lambda_{n-1} = 0$)
then a linear term in $t$ appears, 
$x^{n-1}(t) = e^{\lambda_{n-1}t}(x^{n-1}_0 + t A^{n-1}{}_n \, x^n_0)$.
Knowing $x^{n-1}(t)$ we can now calculate, using again (\ref{5.6}),
$x^{n-2}(t)$ and so on for all the components of $\ve{x}(t)$ in
this basis. 
The final result will be that the components are all sums
of exponentials by polynomials. Indeed note that the integral of
a polynomial of degree $m$ by an exponential is also 
a polynomial of the same degree by the same exponential plus an integration constant
(unless the exponential is zero, in which
case the result is a polynomial of degree $m+1$).
\espa

\ejer: Prove this last statement by induction on the degree
of the polynomial and the use of integration by parts, that is, use
that 
$e^{at}P_m(t) = \frac{d}{dt}(\frac{e^{at}}{a}P_m(t)) - \frac{e^{at}}{a}\frac{d}{dt}{P_m(t)}$
and that  $\frac{d}{dt}{P_m(t)}$ is a polynomial of order $m-1$.

\espa

\noi Rearranging terms we see that the general solution will be 
of the form:

\begin{equation}
  \label{eq:5_sol_gen}
  \ve{x}(t) = \sum_{i=1}^{m} e^{\lambda_i t} \ve{v}_i(t)
\end{equation}
%
where the sum is over the different eigenvalues and where the 
vectors $\ve{v}_i(t)$ are polynomials in $t$.

An important property of the algebraic structure of this solution
is manifested with the following lemma.

\blem
The vector spaces 
\[
E_{\lambda} := \{f(t): \re \mapsto V^{\Complex} | f(t) = e^{\lambda t}P_m(t), 
\;\;\;\mbox{with }\; P_m(t)\;\mbox{a polynomial}\}
\] 
are linearly independent.
\elem

\pru:
We must then prove that if we have a finite sum of elements of
the $E_{\lambda_i}$, $\{\ve{u}_i\}$ and this sum is zero, then each of
these vectors must be identically zero. 
That is
\begin{equation}
  \label{eq:indep_lineal_E_lambda}
  \sum_{i=1}^{m} \ve{u}_i = 0, \;\;\; \ve{u}_i \in E_{\lambda_i}\;\;\; 
  \lambda_i \neq \lambda_j \;\;\; \Rightarrow \ve{u}_i = 0.
\end{equation}

We will prove it by induction on the number of elements in the sum.
The case $m=1$ is trivial. We will then assume that the statement is
true for $m-1$ and we will prove it for $m$.
We will also assume by contradiction that there is a sum of $m$
non-zero elements that is zero, that is:

\begin{equation}
  \label{eq:indep_lineal_E_lambda_2}
  \sum_{i=1}^{m} \ve{u}_i = 0, \;\;\; \ve{u}_i \in E_{\lambda_i},\;\;\; 
  \lambda_i \neq \lambda_j, \;\;\;\ve{u}_i \neq 0.
\end{equation}
%
dividing by $e^{\lambda_1 t}$ we get

\begin{equation}
  \label{eq:indep_lineal_E_lambda_3}
 0 = S(t) := e^{-\lambda_1 t}\sum_{i=1}^{m} \ve{u}_i 
           = P_{m_1}(t) + \sum_{i=2}^{m} e^{(\lambda_i-\lambda_1) t} P_{m_i}(t).
\end{equation}
%
Taking $m_1+1$ derivatives of $S(t)$, where $m_1$ is the order of the polynomial
of the first term we get:

\begin{equation}
  \label{eq:indep_lineal_E_lambda_4}
 0 = \frac{d^{m_1+1}}{dt^{m_1+1}} S(t) 
   =  \sum_{i=2}^{m} e^{(\lambda_i-\lambda_1) t} \tilde{P}_{m_i}(t).
\end{equation}
%
where it is easy to see that the polynomials $\tilde{P}_{m_i}(t)$ are of the same
degree as the originals. We are thus under the inductive hypothesis 
and therefore since these polynomials are non-zero we have a contradiction
\epru
\espa

Let us now see that in fact each of the terms in this sum
is a solution of equation \ref{5.7}. 
Given any $\lambda \in \Complex$ consider the vector space
of functions from $\re \to V^{\Complex}$ of the form $e^{\lambda t} \ve{w}(t)$,
where $\ve{w}(t)$ is a polynomial~\footnote{That is, a linear combination
of vectors in $V^{\Complex}$ with polynomial coefficients in $t$.} in $t$. 
This space is invariant
under the action of $\frac{d}{dt}$, since taking its derivative we get
a similar expression, that is, the product of the same exponential by another
polynomial. 
%
On the other hand it is invariant under the action of
$\ve{A}$, since as $\ve{A}$ does not depend on $t$ its action on any
$e^{\lambda t} \ve{w}(t)$ gives us another similar expression.
%
Therefore, the action of $\frac{d}{dt} -\ve{A}$ will also keep us
in the same space.  
%
Thus we see that if we have a sum of terms with this structure
and with different values of $\lambda$, as is the case in the previous equation
\ref{eq:5_sol_gen}, then if the application of 
$\frac{d}{dt} -\ve{A}$ gives us zero this means that the application of 
$\frac{d}{dt} -\ve{A}$ to each term of
the sum must give zero. That is, each term is a solution of
equation \ref{5.7}!
Thus we conclude that each term in the
 sum \ref{5.7} is of the form $e^{\ve{A}t} \ve{v}_0$ for some
$\ve{v}_0 \in V^{\Complex}$.
%
Let us now consider the subset $W_{\lambda}$ of $V^{\Complex}$, such that
if $\ve{v}_0 \in W_{\lambda}$ then 
$e^{\ve{A}t} \ve{v}_0 = e^{\lambda t} \ve{v}(t)$.
It is clear that the only non-trivial subspaces will be those
with $\lambda = \lambda_i$, where $\{\lambda_i\},\;i=1..d$ are the eigenvalues
of $\ve{A}$. Therefore we must consider only these subsets.
% 
Since this relationship is linear, $W_{\lambda}$ is a subspace of $V^{\Complex}$. 
Since 
$e^{\ve{A}t} \ve{A} \ve{v}_0 = \ve{A} e^{\ve{A}t} \ve{v}_0 
= e^{\lambda t} \ve{A} \ve{v}(t) = e^{\lambda t} \ve{v}'(t)$
and $\ve{v}'(t)$ is also polynomial in $t$, we see
that $W_{\lambda}$ is an invariant subspace of $\ve{A}$.
While using a basis of $W_{\lambda}$ in which the 
restriction to that space of $\ve{A}$ is upper triangular,
we see that $\lambda$ is the only eigenvalue that such a restriction
can have. Finally, since every solution of \ref{5.7} has
the form \ref{eq:5_sol_gen}, the existence theorem of solutions assures us
that any initial data,
that is, any element of $V^{\Complex}$, can be expressed
as a linear combination of elements in $W_{\lambda_i}$. 
Indeed, let $\ve{v}_0$ be any element of $V^{\Complex}$,
from the existence theorem of solutions, we then have 
a unique solution of \ref{5.7} $\ve{x}(t)$ with $\ve{x}(0) = \ve{v}_0$.
But then 
\begin{eqnarray}
  \ve{x}(t) &=& \sum_{i=1}^{m} e^{\lambda_i t} \ve{v}_i(t) \nn
            &=& \sum_{i=1}^{m} e^{A t} \ve{v}_{0i}, 
\end{eqnarray}
for some set of vectors $\{\ve{v}_{0i}\} \in W_{\lambda_i}$.
Evaluating this expression at $t=0$, we obtain
$\ve{v}_0 = \sum_{i=1}^{m} \ve{v}_{0i}$ and we see that the $W_{\lambda_i}$
generate $V^{\Complex}$.
On the other hand, the uniqueness of the solutions implies that no 
element of a given $W_{\lambda_i}$ can be written as
a linear combination of elements of the other $W_{\lambda}$
[otherwise the same initial data would give rise to two
different solutions (since their functional dependence would be
different)]. Indeed, let $0 = \ve{v}_{01}+ \dots + \ve{v}_{0s}$
be any linear combination of elements of $W_{\lambda_i}$, $i=1..s$,
we see that each of them must be zero. By the uniqueness of the solutions
to the ordinary equations, we then have 
$0 = \sum_{i=1}^{m} e^{A t} \ve{v}_{0i} =  \sum_{i=1}^{m} e^{\lambda_i t} \ve{v}_i(t)$
as seen earlier, each element in the last sum must be zero
and therefore, evaluating these at $t=0$, we obtain $\ve{v}_{0i}=0\;\; \forall \; i=1..s$.
Summarizing the above, we have,

\bteo
\label{5_teo_2}
Given an operator $\ve{A}:V^{\Complex} \to V^{\Complex}$, there exists
a set of invariant subspaces of $\ve{A}$, $\{W_{\lambda_i}\}$,
where $ \lambda_i$ are the eigenvalues of $\ve{A}$ such that:
\begin{itemize}
\item[a)] 
$V^{\Complex} = W_{\lambda_1} \oplus W_{\lambda_2}\oplus \cdots W_{\lambda_d}$

\item[b)] 
The only eigenvalue of the restriction of $\ve{A}$ in $\{W_{\lambda_i}\}$
is $ \lambda_i$.
\end{itemize}
\eteo





%%%%%%%%%%%%%%%%%%%%%%%%%%%%%%%%%%%%%%%%%%%%%%%%%%%%%%%%%%%%%%%%%%%%%%%%%

We conclude the study of the solutions of equation \ref{5.7},
that is, of the function $e^{\ve{A}t}$ with a detailed description
of their form obtained using the matrix representation 
of $\ve{A}$ in the basis
where it acquires the Jordan canonical form.
That is, for every invariant subspace $W_{\lambda_i}$ 
the restriction 
of $\ve{A}$ in this subspace has the form,
\beq
\ve{A}= \lap_i + \Delta_i,
\eeq
\noi where the numbers $\lap_i$ are the eigenvalues
corresponding to $\ve{A}$ --generally complex, and therefore we are
now considering $\ve{A}$ as an operator from $C^n$ to $C^n$-- and
$\Delta_i$ is a matrix whose only non-zero components are ones and
can only be in the immediate upper diagonal. The
matrix $\Delta_i$ has the important property of being nilpotent, that is,
there exists an integer $m_i$ less than or equal to the multiplicity of $\lambda_i$, 
such that $\Delta_i^{m_{i}}=0$.

Since $\lap_i I $ and $\Delta_i$ commute, we have that
\beq
e^{t\lambda_iI + t \Delta_i} = e^{t\lambda_i I}e^{t\Delta_i}, 
\eeq
 but $\dip{e^{t\lambda_i I}= e^{t \lambda_i}I}$
 and $ e^{t\Delta_i}=\dip\sum_{j=0}^{m_i-1}\;\frac{(t\Delta_i)^j}{j!}$,
that is, a finite sum. 
\espa
Since the exponential of $\ve{A}$ is a sum of powers of $\ve{A}$, it follows that the
invariant spaces of this are the same as those of $\ve{A}$ and therefore, 
$e^{t\ve{A}}|_{B_i} = e^{t\ve{A}_i}$. 
From this, we conclude that the matrix $e^{t\ve{A}}$ is of the form
\beq
e^{t\ve{A}}=\left(\barr{cccc}
             k_0 & & &  \\
              & k_1 & &  \\
              & & \ddots &  \\
              & & & k_p 
              \earr\right)
\eeq
\noi with each $k_i$ a square sub-matrix, the $k_0=diag(e^{t \lap_1},
\ldots,e^{ t\lap_n})$ and if $i\neq 0$
\beq
k_i=e^{t\lap_i}\left[\barr{cccccc}
 1 & t & t^2/2 & \cdots & \cdots & t^{n-1}/(n-1)! \\
   & 1 & t & \cdots & \cdots & \vdots  \\
   &   & 1 & \cdots & \cdots & \vdots  \\
   &  &   & \ddots  & \cdots &t^2/2  \\
   &  0 & &  & \ddots & t  \\
   & & & & & 1  \\
\earr\right],
\eeq

\noi where the number of rows or columns is the same as those of the
$J_i$ corresponding to $\ve{A}$ in the Jordan composition, this is
less than or equal to the multiplicity of $\lap_i$ as a root of the characteristic polynomial.

The basis in which $\ve{A}$ has the Jordan canonical form is generally
complex, it is actually a basis in $C^n$. If we wish to use a real basis,
which is possible since we start with $\ve{A}$ as an operator
from $\re^n$ to $\re^n$, and we make the corresponding transformation,
the matrix $e^{\ve{A}t}$ will undergo the corresponding similarity transformation.
Since this transformation is independent of
time, although the components of $e^{\ve{A}t}$ will not have the previous form,
they will be sums of exponential terms by
polynomials in $t$. Therefore, each component of the general solution
of \ref{5.7} will be of the form,
\beq
x^i(t) = \sum_{p=1}^{q} \left\{(e^{t \lambda_p}+e^{t \bar{\lambda}_p}) \,P^i_p(t) +
i(e^{t \lambda_p}-e^{t\bar{\lambda}_p})\,Q^i_p(t)\right\}
\label{5.8}
\eeq

\noi
where $q$ is the number of distinct eigenvalues --counting complex pairs as one-- and $P^i_p\,,\;Q^i_p$ are polynomials in
$t$ whose degree is less than or equal to the multiplicity with which the
eigenvalue $\lap_p$ appears as a root of the characteristic polynomial. This
information is useful in two ways. Practically, because
although the method of constructing the solution using a basis in
which $\ve{A}$ has the Jordan canonical form is straightforward, for
large dimension systems it becomes cumbersome. In some cases, it is
convenient to calculate the eigenvalues and their multiplicity and then
assume a solution of the form \ref{5.8} and calculate the polynomials 
$P^i_p\,,\;Q^i_p$.

This method is also useful in the sense that it allows us to know the
global behavior of the solutions. For example, we see that if the
real part of the eigenvalues is not positive and those whose real
part is zero do not appear repeated, then all solutions are
bounded [There exists $C>0$ such that $\|\ve{x}(t)\|<C$.] If in addition, all have
negative real part, then all solutions tend
asymptotically to the trivial solution [$\dip\lim_{t\to\infty}
\ve{x}(t) = 0$ ].


%************************************************************************
We will now analyze in detail the case where all eigenvalues are distinct. Note that although this is the generic case --in the sense that if a system has coincident eigenvalues, then there are arbitrarily small modifications to it that make the eigenvalues distinct-- systems with coincident eigenvalues do appear in physics. This case also contemplates the situation of a general system where the initial data belongs to one of the one-dimensional subspaces $B_i$.

Since the operator $\ve{A}$ is real, its eigenvalues will be real or complex conjugates [if $det(\ve{A}-\lap I)=0$ then $det\overline{(\ve{A}-\lap I)}=det(\ve{A}-\bar{\lap} I)=0$]. If $\lap_i$ is real, then its eigenvector can be chosen to be real. 
Indeed, if $(\ve{A}-\lap I)\ve{u}_i=0$, then also 
$(\ve{A}-\lap I)\bar{\ve{u}}_i=0$, 
but the roots are simple and therefore each $\lap_i$ has only one eigenvector --modulo a complex scalar--, that is 
$\bar{\ve{u}}_i=\alpha\,\ve{u}_i$ $\alpha\in C$. 
Choosing 
$\ve{v}_i=\ve{u}_i+\bar{\ve{u}}_i=(1+\alpha)\,\ve{u}_i$ we obtain a real eigenvector.

In this case, the component $x^i_0$ of $\ve{x}_0$ in the eigenbasis in the direction $\ve{v}_i$ evolves as
\beq
x^i(t)=x^i_0\,e^{\lap_it},
\eeq

\noi
that is, it grows or decays exponentially with time according to the sign of $\lap_i$.

If the eigenvalue $\lap_i$ is complex, then we can choose its eigenvector $\ve{u}_i$ such that it is the complex conjugate of the chosen one corresponding to $\bar{\lap}_i$. This pair of eigenvectors generates a 2-dimensional complex subspace. If $\ve{x}_0$ belongs to this subspace and is real, then it will have the form 
$\ve{x}_0=a(\ve{u}_i+\bar{\ve{u}}_i)-ib(\ve{u}_i-\bar{\ve{u}}_i)$ 
with $a$ and $b$ real, that is 
$\ve{x}_1=\ve{u}_i+\bar{\ve{u}}_i$ and $\ve{x}_2=i(\ve{u}_i-\bar{\ve{u}}_i)$ 
form a real basis.

How do these vectors change if we apply the operator
$e^{\ve{A}t}$ to them? That is, what are the solutions of the equation 
$\dot{\ve{x}}=\ve{A}\,\ve{x}$ 
with initial conditions $\ve{x}_1$ and $\ve{x}_2$? 
Calling these
$\ve{x}_1(t)$ and $\ve{x}_2(t)$ respectively and using that
$e^{\ve{A}t}\,\ve{u}_i=e^{\lap_it}\,\ve{u}_i$ we obtain,
\beq \barr{rcl}
\ve{x}_1(t) &=  &e^{\alpha_i t}(\ve{x}_1\cos w_it -\ve{x}_2\sin w_it) \\
\ve{x}_2(t) &=  &e^{\alpha_it}(\ve{x}_2\cos w_it + \ve{x}_1 \sin w_it),
\earr
\eeq
\noi with $\lap_i=\alpha_i+i\,w_i$.

We see that the action of the operator $e^{\ve{A}t}$ in this case is to dilate the vectors by a factor of $e^{\alpha_it}$ and to rotate them by an angle $w_i t$.

%%%%%%%%%%%%%%%%%%%%%%%%%%%%%%%%%%%%%%%%%%%%%%%%%%%%%%%%%%%%%%%%%%%%%%%%%%%%%

%\newpage
\section{Problems}

%%%%%%%%%%%%%%%%%%%%%%%%%%%%%%%%%%%%%%%%%%%%%%%%%%%%%%%%%%%%%%%%%%%%%%%%%%%%%


%%%%%%%%%%%%%%%%%%%%%
\bpro
Given a set of functions $\{f_i(t)\}, \;i=1,..,n$, vectors are defined as
$\ve{u}_i(t) := (f_i(t), f^{(1)}_i(t),\dots f^{(n-1)})$ and the
Wronskian of the system as 
$W(\{f_i\})(t) := \eps(\ve{u}_1(t),\ve{u}_2(t),\dots, \ve{u}_n(t))$.
If the Wronskian of a set of functions does not vanish, then the
functions are linearly independent, that is, no non-trivial linear combination of them (with constant coefficients) vanishes.
The converse is not true.
Calculate the Wronskian of the following sets:

a) $\{4,t\}$

b) $\{t,3t,t^2\}$

c) $\{e^t, te^t, t^2e^t\}$

d) $\{ \sin(t), \cos(t), \cos(2t) \}$

e) $\{1, \sin(t), \sin(2t)\}$
\epro

\bpro
Decide if the following set of functions is linearly dependent or not. Then calculate the Wronskian.

\begin{equation}
  f_1(t) = \left\{
    \begin{array}{ll}
      0, & 0 \leq x \leq 1/2 \\
      (x-1/2)^2, & 1/2 \leq x \leq 1
    \end{array}
    \right.
\end{equation}

\begin{equation}
  f_2(t) = \left\{
    \begin{array}{ll}
      (x-1/2)^2, & 0 \leq x \leq 1/2 \\
      0, & 1/2 \leq x \leq 1
    \end{array}
    \right.
\end{equation}
\epro
%%%%%%%%%%%%%%%%%%%%%

\bpro
Use the theory of ordinary differential equations to prove the following identities:

a)
\begin{equation}
  e^{s\ve{A}} e^{t\ve{A}} = e^{(s+t)\ve{A}},
\end{equation}
 
b)
\begin{equation}
  e^{\ve{A}} e^{\ve{B}} = e^{\ve{A}+\ve{B}},\;\;\; 
  \mbox{if and only if} \;\;[\ve{A},\ve{B}] := \ve{A}\ve{B} - \ve{B}\ve{A}= 0.
\end{equation}
\epro

%%%%%%%%%%%%%%%%%%%%%

\bpro
Graph the vector fields corresponding to the following systems and some typical sets of solutions.

a) 
\begin{equation}
  \label{eq:prob5_2a}
  \frac{d}{dt}\left(
    \begin{array}{c}
      x_1 \\ x_2
    \end{array}
  \right) = 
    \left(
      \begin{array}{cc}
        1 & 0 \\ 0 & \frac{1}{2}
      \end{array}
    \right) \left(
      \begin{array}{c}
        x_1 \\ x_2
      \end{array}
      \right)
\end{equation}

b) 
\begin{equation}
  \label{eq:prob5_2b}
  \frac{d}{dt}\left(
    \begin{array}{c}
      x_1 \\ x_2
    \end{array}
  \right) = 
    \left(
      \begin{array}{cc}
        1 & 0 \\ 0 & -1
      \end{array}
    \right) \left(
      \begin{array}{c}
        x_1 \\ x_2
      \end{array}
      \right)
\end{equation}

c) 
\begin{equation}
  \label{eq:prob5_2c}
  \frac{d}{dt}\left(
    \begin{array}{c}
      x_1 \\ x_2
    \end{array}
  \right) = 
    \left(
      \begin{array}{cc}
        1 & 1 \\ 0 & 1
      \end{array}
    \right) \left(
      \begin{array}{c}
        x_1 \\ x_2
      \end{array}
      \right)
\end{equation}

d) 
\begin{equation}
  \label{eq:prob5_2d}
  \frac{d}{dt}\left(
    \begin{array}{c}
      x_1 \\ x_2
    \end{array}
  \right) = 
    \left(
      \begin{array}{cc}
        -1 & 0 \\ 1 & -1
      \end{array}
    \right) \left(
      \begin{array}{c}
        x_1 \\ x_2
      \end{array}
      \right)
\end{equation}

e) 
\begin{equation}
  \label{eq:prob5_2e}
  \frac{d}{dt}\left(
    \begin{array}{c}
      x_1 \\ x_2
    \end{array}
  \right) = 
    \left(
      \begin{array}{cc}
        0 & 1 \\ 1 & 0
      \end{array}
    \right) \left(
      \begin{array}{c}
        x_1 \\ x_2
      \end{array}
      \right)
\end{equation}

f) (compare the solutions with those of point b)
\begin{equation}
  \label{eq:prob5_2f}
  \frac{d}{dt}\left(
    \begin{array}{c}
      x_1 \\ x_2
    \end{array}
  \right) = 
    \left(
      \begin{array}{cc}
        0 & -1 \\ 1 & 0
      \end{array}
    \right) \left(
      \begin{array}{c}
        x_1 \\ x_2
      \end{array}
      \right)
\end{equation}

g) 
\begin{equation}
  \label{eq:prob5_2g}
  \frac{d}{dt}\left(
    \begin{array}{c}
      x_1 \\ x_2
    \end{array}
  \right) = 
    \left(
      \begin{array}{cc}
        i & 0 \\ 0 & i
      \end{array}
    \right) \left(
      \begin{array}{c}
        x_1 \\ x_2
      \end{array}
      \right)
\end{equation}

h) 
\begin{equation}
  \label{eq:prob5_2h}
  \frac{d}{dt}\left(
    \begin{array}{c}
      x_1 \\ x_2
    \end{array}
  \right) = 
    \left(
      \begin{array}{cc}
        1+i & 0 \\ 0 & 1-i
      \end{array}
    \right) \left(
      \begin{array}{c}
        x_1 \\ x_2
      \end{array}
      \right)
\end{equation}

i) 
\begin{equation}
  \label{eq:prob5_2i}
  \frac{d}{dt}\left(
    \begin{array}{c}
      x_1 \\ x_2
    \end{array}
  \right) = 
    \left(
      \begin{array}{cc}
        1+i & 0 \\ 1 & 1-i
      \end{array}
    \right) \left(
      \begin{array}{c}
        x_1 \\ x_2
      \end{array}
      \right)
\end{equation}


\epro

\bpro
Reduce these equations to first-order systems and find the general solution of the equations:

a) $\frac{d^3 x}{dt^3} - 2\frac{d^2 x}{dt^2} - 3\frac{d x}{dt} =0$

b) $\frac{d^3 x}{dt^3} + 2\frac{d^2 x}{dt^2} + \frac{d x}{dt} =0$

c) $\frac{d^3 x}{dt^3} + 4\frac{d^2 x}{dt^2} + 13 \frac{d x}{dt} =0$

d) $\frac{d^4 x}{dt^4} + 4\frac{d^3 x}{dt^3} + 8\frac{d^2 x}{dt^2} + 8 \frac{d x}{dt} + 4 = 0$
\epro

\bpro[Newton's Law]
Consider the equation $\frac{d^2 x}{dt^2} + f(x) = 0$.

a) Prove that $\frac{1}{2} \dot{x}^2 + \int_{x_0}^x f(s)ds$ is a first integral.

b) Find the first integral of $\frac{d^2 x}{dt^2} - x + x^2/2=0$.

c) Graph the corresponding vector field and some of its solutions.
Find its stationary solutions (or equilibrium points) and study their
neighborhoods by linearizing the equation at these points.
\epro

\bpro
Study the following system
\begin{eqnarray}
  \dot{x}_1 &=& x_1 - x_1x_2 - x_2^3 + x_3(x_1^2 + x_2^2 -1 - x_1 +x_1x_2+ x_2^3) \nn
  \dot{x}_2 &=& x_1 - x_3(x_1 - x_2 + x_1 x_2) \nn
  \dot{x}_3 &=& (x_3 -1)(x_3+2x_3x_2^2 + x_3^3)
\end{eqnarray}

a) Find the equilibrium points.

b) Show that the planes $x_3=0$ and $x_3 = 1$ are invariant sets,
(that is, solutions that start in them never leave them).

c) Consider the invariant set $x_3 = 1$ and see if it has periodic solutions.
\epro







%%% Local Variables: 
%%% mode: latex
%%% TeX-master: "apu_tot"
%%% End:
