\chapter*{Preface}

These notes, now turned into a book, originated as an attempt to condense in one place a large set of ideas, concepts, and mathematical tools that I consider basic for the understanding and daily work of a physicist today.

It usually happens that if a problem is formulated from a physical need, such as the description of some natural phenomenon, then it is well formulated, in the sense that a reasonable solution to it exists. This rule has generally been very fruitful and has particularly served as a guide to many mathematicians to make their way in unknown areas. But it has also served, particularly to many physicists, to work without worrying too much about formal aspects, whether analytical, algebraic, or geometric, and thus be able to concentrate on physical and/or computational aspects. Although this allows for the rapid development of some research, in the long run, it leads to stagnation because by proceeding in this way, one avoids facing problems that are very rich in terms of conceptualizing the phenomenon to be described. It is important to verify that the formulated problem has a mathematically and physically correct solution.

An example of this was the development, in the middle of the last century, of the modern theory of partial differential equations. Many of these equations arose because they describe physical phenomena: heat transmission, electromagnetic wave propagation, quantum waves, gravitation, etc. One of the first mathematical responses to the development of these areas was the Cauchy-Kowalevski theorem, which tells us that given a partial differential equation (under quite general circumstances), if an analytic function is given as data on a hypersurface (with certain characteristics), then there is a unique solution in a sufficiently small neighborhood of that hypersurface. It took a long time to realize that this theorem was not really relevant from the point of view of physical applications: there were equations admitted by the theorem such that if the data was not analytic, there was no solution! And in many cases, if they existed, they did not depend continuously on the given data, a small variation of the data produced a completely different solution. It was only in the middle of the last century that substantial progress was made on the problem, finding that there were different types of equations, hyperbolic, elliptic, parabolic, etc., that behaved differently and this reflected the different physical processes they simulated. Due to its relative novelty, this very important set of concepts is not part of the set of tools that many active physicists have, nor are they found in the textbooks usually used in undergraduate courses.

Like the previous one, there are many examples, particularly the theory of ordinary differential equations and geometry, without which it is impossible to understand many of the modern theories, such as relativity, elementary particle theories, and many phenomena of solid-state physics. As our understanding of the basic phenomena of nature advances, we realize that the most important tool for their description is geometry. This, among other things, allows us to handle a wide range of processes and theories with little in common with each other with a very reduced set of concepts, thus achieving a synthesis. These syntheses are what allow us to acquire new knowledge, since by adopting them we leave space in our minds to learn new concepts, which are in turn ordered more efficiently within our mental construction of the area.

These notes were originally intended for a four-month course. But in reality, they were more suited for an annual course or two semesters. Then, as more topics were incorporated into them, it became increasingly clear that they should be given in two semesters or one annual course. Basically, one course should contain the first chapters that include notions of topology, vector spaces, linear algebra, ending with the theory of ordinary differential equations. The task is considerably simplified if the students have previously had a good course in linear algebra. The correlation with physics subjects should be such that the course is prior to or concurrent with an advanced mechanics course. Emphasizing in it the fact that ultimately one is solving ordinary differential equations with a certain special structure. Using the concepts of linear algebra to find eigenmodes and the stability of equilibrium points. And finally using geometry to describe, albeit briefly, the underlying symplectic structure.

The second course consists of developing the tools to be able to discuss aspects of the theory of partial differential equations. It should be given before or concurrently with an advanced electromagnetism course, where emphasis should be placed on the type of equations that are solved (elliptic, hyperbolic), and the meaning of their initial or boundary conditions, as appropriate. Also using coherently the concept of distribution, which is far from being an abstract mathematical concept but is actually a concept that naturally appears in physics.

None of the content of these notes is original material, but some ways of presenting it are, for example, some simpler proofs than usual, or the way of integrating each content with the previous ones. Much of the material should be thought of as a first reading or an introduction to the topic and the interested reader should read the cited books, from which I have extracted much material, being these excellent and difficult to surpass.
