% !TEX encoding = IsoLatin
% !TEX root =  ../Current_garamond/libro_gar.tex

%%ultima modificación 28/05/2013


\chapter{Symmetric--Hyperbolic Equations}

\section{Introduction}

In this chapter we will study systems of hyperbolic equations
under the following restriction:
\espa

\defi: We will say that a system is {\bf symmetric--hyperbolic} if:

\begin{itemize}
\item[a.)] The target space of the linear map $M^a_{A'B}k_a$ is of
the same dimension as the domain space for all $k_a \neq 0$.
Therefore, from now on we will use unprimed indices.

\item[b.)] The map $M^a_{AB}k_a$ is symmetric for all $k_a \neq 0$.

\item[c.)] At each point of the manifold there exists a co-vector $t_{a}$ such that the map,
$H_{AB} := M^a_{AB}t_a$ is positive definite. (That is,
$H_{AB}u^Au^B \geq 0$ ($= 0\; \mbox{iff} \; u^A = 0$).)~\footnote{Since the set of symmetric and positive maps is open, if $x_{a}$ is any other co-vector, then $H_{AB}(\eps) := M^a_{AB}(t_a+\eps x_{a})$ will also be positive for $\eps$ sufficiently close to zero. Therefore, level surfaces $\tau = const.$ can be locally defined such that $H_{AB} := M^a_{AB}\nabla_{a}\tau$ is positive on the entire local surface.}

%In a neighborhood of each point there exists a function $\tau$
%such that if we call its differential $t_a$ ($=\na_a \tau$), the
%map  $H_{AB} := M^a_{AB}t_a$ is positive definite. (That is,
%$H_{AB}u^Au^B \geq 0$ ($= 0\; \mbox{iff} \; u^A = 0$).)

\end{itemize}
Note that this last condition implies that $H_{AB}$ is a
metric in the space of independent variables. This and its inverse,
which we will denote by $H^{AB}$, will be used to raise and lower indices.

This is also not a significant restriction from the point of view
of physics, since all physical systems we know of
are symmetric--hyperbolic.

Also, but only for simplicity in the exposition as
this will avoid some technical complications, we will only consider
linear systems.

We will begin this chapter with a simple example that illustrates the
basic characteristics of this class of equations.

\section{An Example}

Consider an infinite string in the $x,y$ plane and let $y = u(x,t)$ be the position
of the string at time $t$ in that plane.
By adjusting the dimensions (of length or time) it can be seen that
$u(x,t)$ satisfies the equation,
\beq
 -\frac{\pa^2u}{\pa t^2} + \frac{\pa^2 u}{\pa x^2} = f(x,t),
\label{1)}
\eeq
where $f(x,t)$ is the force density exerted on the
string and which
we assume does not depend on the position of the string with respect to the
$y$ coordinate.\footnote{Otherwise we would have to consider $f(x,t,u)$ which
complicates the problem.}
We thus have the mathematical problem of finding the solutions to the equation,
\beq 
\Box u = g^{ab} \na_a \na_b u = f,
\eeq
in $M = \re^2$ with pseudo-Euclidean metric $g_{ab} = -(dt)^2 + (dx)^2.$
To handle this equation it is convenient to introduce a coordinate system
appropriate to this metric, that is, one that has its characteristic lines as axes,
\beq\barr{rcl}
\xi &=& x + t = const.,\\
\eta& =& x - t = const.
\earr
\eeq

\espa 
%\fig{6cm}{Null Coordinate System}
\begin{figure}[htbp]
  \begin{center}
    \resizebox{7cm}{!}{\myinput{Figure/m13_1}}
    \caption{Null coordinate system.}
    \label{fig:13_1}
  \end{center}
\end{figure}

we then have that
\beq\barr{rcl}
x &=& \frac{\xi + \eta}2\\
t &=& \frac{\xi - \eta}2
\earr
\eeq
and
\beq\barr{rcl}
g_{ab} = -(dt)^2 + (dx)^2 &=& \frac14[\{-(d\xi)^2 - (d\eta)^2 + d\xi
\otimes d\eta + d\eta \otimes d\xi \}\\
& & + \{(d\xi)^2 + (d\eta)^2 +
d\xi\otimes d\eta + d\eta\otimes d\xi\}]\\
&=& \frac12[d\xi \otimes d \eta + d \eta \otimes d \xi]
\earr
\eeq
Noting that $g^{ab} = 2[\pa \xi \otimes \pa \eta + 
\pa \eta \otimes \pa \xi]$
and that the Christoffel symbols vanish because the metric has
constant components we have,\footnote{
Note also that $g(\pa_{\eta},\pa_{\eta}) = g(\pa\xi,\pa \xi) = 0$, that is, these coordinate vectors
have null norm.}
\beq \Box u = 4\frac{\pa^2u}{\pa \eta \pa \xi} = f(\eta,\xi).
\eeq
This equation can be integrated immediately, obtaining
\beq\barr{rcl}
\dip\frac {\pa u}{\pa \eta}(\eta,\xi)&=& \frac{1}{4}\dip\int^{\xi}_{\xi_0} f(\eta,\ti \xi)
\;d\ti \xi \;+ C(\eta)\\ [3mm]
u(\eta,\xi) &=& \frac{1}{4}\dip\int^{\eta}_{\eta_0}\dip\int^{\xi}_{\xi_0} f(\ti \eta,\ti \xi)
d\ti \xi d\ti \eta + u_I(\xi) + u_{II} (\eta),
\earr
\eeq
where $u_I(\xi)$ and $u_{II} (\eta)$ are arbitrary functions.
Let us first consider the case $f \equiv 0$, that is, the homogeneous equation. 
Its solutions are then the sum of any function of $\xi$ 
and any function of $\eta$.
Returning to the $x,t$ coordinates we obtain
\beq 
u(x,t) = u_I(x+t) + u_{II}(x-t). 
\eeq
For example,
\beq
u_I(x+t) = \lb\barr{cl}e^{\frac1{(x+t)^2 - 1}}& x+t \in [-1,1]\\
            0               &            x+t \in (-\infty,-1]\cup [1,+\infty)
\earr\right.
\eeq          
is a solution that represents a {\bf wave}
($\equiv$ solution of the homogeneous equation) moving to the left without changing shape and with
speed 1.

\espa 
%\fig{6cm}{Wave Propagation.} 
\begin{figure}[htbp]
  \begin{center}
    \resizebox{7cm}{!}{\myinput{Figure/m13_2}}
    \caption{Wave propagation.}
    \label{fig:13_2}
  \end{center}
\end{figure}

Similarly $u_{II}$ represents a wave moving to the right.
See figure.
Let us now see that the Cauchy problem in this case has a solution.
This is extremely important in physics: it tells us that if we take
a non-characteristic surface (for example $t=0$) and give there
as data $u$ and its time derivative we will obtain a unique solution for
future times. 
This is what allows us, if we know the present,
to predict the future, that is, if we prepare an experiment, to predict
the result. This fact is what distinguishes physics from
other natural sciences.
Suppose then that at $t=0$ $(\xi = \eta = x)$ we give $u(x,0) = u_0(x)$ 
and its derivative
$\frac{\pa u}{\pa t}(x,0) = u_1(x)$. 
We then have that
\beq
u_0(x) = u(x,0) = u_I(x) + u_{II}(x),
\label{2)}
\eeq
\beq
 u_1(x) = \frac{\pa u}{\pa t} (x,0) = u'_I (x) - u'_{II}(x).
\label{3)}
\eeq
Differentiating \ron{2)} with respect to $x$ and solving the linear system thus
obtained we have,
\beq\barr{rcl}
u'_I(x) &=& \dip\frac{u'_0(x) + u_1(x)}2\\
u'_{II} &=& \dip\frac{u'_0(x) - u_1(x)}2,
\earr
\eeq
and integrating,
\beq\barr{rcl}
u_I(x) &=& \dip\frac{u_0(x)}2  + \dip\frac12 \dip\int^x_0 u_1(\ti x) d\ti x +
C_I,\\ [3mm]
u_{II}(x) &=& \dip\frac{u_0(x)}2  - \dip\frac12 \dip\int^x_0 u_1(\ti x) d\ti x 
+ C_{II}.
\earr
\eeq
For \ron{2)} to be satisfied we must have $C_I=-C_{II}$ and therefore
\beq
 u(x,t) = \frac2 (u_0(x+t) + u_0(x-t)) + \frac2 \int^{x+t}_{x-t}
u_1(\ti x) d\ti x.
\label{4)}
\eeq
We then see that if we give as data $u_0(x) \in C^2(\re)$ and $u_1(x)
\in C^1(\re)$ we obtain a
solution $u(x,t) \in C^2(\re \times \re)$.
By construction this solution is unique.

\espa
\ejer:
Explicitly show that \ron{4)} satisfies \ron{1)} 
with $f \equiv 0$.

\espa
\ejer: 
Use the general solution found earlier to see that the homogeneous solutions of the wave equation in two dimensions satisfy $u(t,x) = u(0,x+t) + u(0,x-t) - u(-t,x)$. 
\espa

Equation \ron{4)} tells us that $u(x,t)$ is contributed \textit{only} 
by the average of the
values of $u_0$ at $x-t$ and $x+t$ and the integral of $u_1$, between these
two values. [See figure 13.3.]
%
\espa 
%\fig{6cm}{General Homogeneous Solution in Two Dimensions.} 
\begin{figure}[htbp]
  \begin{center}
    \resizebox{7cm}{!}{\myinput{Figure/m13_3}}
    \caption{General homogeneous solution in 1+1 dimensions.}
    \label{fig:13_3}
  \end{center}
\end{figure}

%
What happens if we have a source $f(x,t)$? Since we already have the general solution (for arbitrary Cauchy data) of the homogeneous equation, we only need the solution of the inhomogeneous equation with zero data. 
This is achieved by integrating $f(\xi, \eta)$ first with respect to $\ti \xi$ 
between $\ti \eta$ and $\xi$ and
then $\ti \eta $ between $\xi$ and $\eta$.
\beq
v(\eta,\xi) = \frac{1}{4} \int^{\eta}_{\xi}\lp\int^{\xi}_{\ti \eta} f(\ti \xi, \ti
\eta) d \ti \xi \rp d\ti \eta.
\eeq

\espa 
%\fig{6cm}{General Inhomogeneous Solution.} 
\begin{figure}[htbp]
  \begin{center}
    \resizebox{7cm}{!}{\myinput{Figure/m13_4}}
    \caption{General inhomogeneous solution.}
    \label{fig:13_4}
  \end{center}
\end{figure}

\espa

\ejer:
Show that 
\beq
v(x,t) = \int^t_0\lp\int^{x+(t-\ti t)}_{x-(t-\ti t)} f(\ti x, \ti
t) d \ti x \rp d\ti t,
\eeq
that
\beq v(x,t)|_{t=0} = \frac{\pa v}{\pa t} (x,t)|_{t=0} = 0,\eeq
and that $v(x,t)$ satisfies \ron{1)}.
\espa

\noi We then see that the solution sought is,
{\small
\beq
u(x,t) = \frac2 (u_0(x+t) + u_0(x-t)) + \frac2 \int^{x+t}_{x-t}
u_1(\ti x) d\ti x +
\int^t_0\lp\int^{x+(t-\ti t)}_{x-(t-\ti t)} f(\ti x, \ti
t) d \ti x \rp d\ti t,
\eeq
}

\noi which by construction is unique and that $u(x_0,t_0)$ depends on the initial values and on $f$ in the conical region with vertex $(x_0,t_0)$ given by,

\beq
\lb\barr{l}
t \leq t_0\\
|x-x_0| \leq t_0 -t.
\earr\right.
\eeq
This region is called the {\bf domain of dependence} of the point $(x_0,t_0)$, 
only what happens in this
region can affect the value of $u$ at that point.
Similarly, the {\bf domain of influence} of a point $(x_0,t_0)$ is defined
as the set of
points where the value of $u$ can be changed if the
value of $u$, its derivative, or $f$ is changed at $(x_0,t_0)$.
In this case, this is given by: $\{(x,t) | t \geq t_0, |x-x_0| \leq
t - t_0 \}$.

The behavior of the solutions of hyperbolic equations
is generally the same as that of this simple example: Given generic Cauchy data there will be a unique solution (both to the
future and to the past). In the case of linear equations
this solution can be extended indefinitely in both
temporal directions, in the nonlinear case the solutions are only valid in a finite temporal interval and it is an interesting
physical problem to see if the physical nonlinear equations
can or cannot be extended indefinitely and what it means
physically the appearance of singularities in the
solutions.~\footnote{A singularity is a point where $u$ ceases to be
sufficiently differentiable for the equation to make
sense or worse where $u$ ceases to make sense even as
a distribution.} 
A quantity of great physical and mathematical importance related
to the wave equation is the energy of the solutions. In two
dimensions this is given by,
\beq
E(u,t_0) = \frac2 \int_{t=t_0} [(\frac{\pa u}{\pa t})^2 + (\frac{\pa
u}{\pa x})^2 ] dx.
\eeq
Observe that the energy is positive and its rate of change is
given by,
\beq\barr{rcl}
\dip \frac{dE}{dt}(u,t_0) &=&
\dip\int_{t=t_0} (\frac{\pa u}{\pa t} \frac{\pa^2 u}{\pa t^2} +
\dip\frac{\pa^2 u}{\pa t \pa x} \frac{\pa u}{\pa x}) dx\\ [3mm]
&=&
\dip\int_{t=t_0} [\frac{\pa u}{\pa t} (\frac{\pa^2 u}{\pa t^2} -
\dip\frac{\pa^2 u}{\pa x^2}) +
\dip\frac{\pa }{\pa x}(\frac{\pa u}{\pa t} \frac{\pa u}{\pa x})] dx\\ [3mm]
&=& 
\dip\int_{t=t_0} [\frac{\pa u}{\pa t} f ] dx,
\earr
\eeq
where in the last equality we have used \ron{1)} and assumed that 
$\dip\lim_{x \to \infty} \dip\frac{\pa u}{\pa t} \frac{\pa u}{\pa x} = 0$.
If $f=0$ then the energy is conserved and this gives us an alternative proof
of the uniqueness of the solutions.

\bteo[Uniqueness] At most, there exists a unique solution $u(x,t)$ to the wave equation among the functions in $u(x,t) \in H^1(\re)$, $\frac{\pa u}{\pa t} (x,t) \in H^0(\re)$ (where $\re$ is the surface $t = const$) for a given Cauchy data. \eteo

\espa

\pru: Suppose there are two solutions $u_1$ and $u_2$ with the same Cauchy data at, say $t = 0$. Then $\delta u = u_1 - u_2$ satisfies the homogeneous equation and has zero Cauchy data. Therefore $E(\delta u,t=0) = 0$, but the energy of $\delta u$ is conserved and thus $E(\delta u,t-t_0) = 0 ;;; \forall t_0$. This implies that $|\frac{\pa \delta u}{\pa t} (x,t)|{t=t_0} |{H^0(R)} = 0$ and therefore $\frac{\pa \delta u}{\pa t} (x,t) |{t=t_0} = 0$ at almost every point. Similarly, we have that $|\frac{\pa \delta u}{\pa x} (x,t)|{t=t_0} |{H^0(R)} = 0$ and therefore $\frac{\pa \delta u}{\pa x} (x,t) |{t=t_0} = 0$ or $\delta u = const$. But constants are not square integrable in \re{} and therefore $\delta u(x,t) = u_1(x,t) - u_2(x,t) = 0$ as an element of $H^1(\re)$ $\spadesuit$

%%%%%%%%%%%%%%%%%%%%%%%%%%%%%%%%%%%%%%%%%%%%%%%%%%%%%%%%%%%%%%%%%%%%%%%%%%%% \section{Energy inequality for symmetric-hyperbolic systems}

In this section, we will consider a general linear symmetric-hyperbolic system. That is, a system of the form: \beq M^a_{AB} \na_a u^B = I_A, \label{en1} \eeq with $M^a_{AB}$ and $I_A$ generally dependent on position, with $M^a_{AB}$ symmetric in the uppercase indices and such that there exists a function $\tau$ with gradient $t_a$ such that $H_{AB}$ is positive definite and therefore invertible.

Let $\Sigma_t$ be the family of surfaces given by the level surfaces $\tau = t$. Let $\Gamma$ be any region of $\Sigma_0$ and let $\Omega$ be a region such that $\Omega \cap \Sigma_0= \Gamma$ and also $\Gamma(t) = \Omega \cap \Sigma_t$.

Let $u^A$ be a solution of \ref{en1} and let,

\beq E(u^A,t) = \int_{\Gamma(t)} n_aM^a_{AB}u^Au^B, \eeq that is, the integral of $H_{AB}u^Au^B$ over a region of the hypersurface $\tau=t=const.$, where we have defined $H_{AB}$ using $n_a = t_a/|t_a|$, that is, we have normalized $t_a$.

Let us see the difference of this quantity between two surfaces as shown in the figure.

\espa % %\fig{6cm}{Energy inequality} % 
\begin{figure}[htbp] 
    \begin{center} 
    \resizebox{7cm}{!}{\myinput{Figure/m13_5b}} 
        \caption{Energy inequality} 
        \label{fig:13_5b} 
    \end{center}
\end{figure}
\espa

\begin{figure}[htbp] 
    \begin{center} \resizebox{7cm}{!}{\myinput{Figure/m13_5}} 
    \caption{Energy inequality, perspective view.} 
    \label{fig:13_5} 
    \end{center}
\end{figure}

To do this, we will use the divergence theorem, which tells us that, 
\beq 
E(u^A,t) - E(u^A,0) + E(u^A,s) = \int_{\Omega(0,t)} \na_a( M^a_{AB} u^Au^B), 
\eeq 
where the negative sign of the second term on the left is due to the fact that in the definition of $E(u^A,0)$ 
we take the inward normal to the region $\Omega(0,t)$. 
The last term represents the integral of the energy over a surface $S$ which we will assume is given by the 
equation $\sigma = s$ where $\sigma$ is a smooth function and $s \in \re$. 
This term represents the energy that escapes from the region through the surface $S$.

Using equation \ref{en1} on the right-hand side we get, 

\begin{eqnarray}
    \!\!\!\!\!\!E(u^A,t) - E(u^A,0) + E(u^A,s) &=& 
    \int_{\Omega(0,t)} [(\na_a M^a_{AB}) u^Au^B + 2I_Au^A] \\ 
    %
    &\leq& \int^t_0 \int_{\Sigma_{\tilde{t}}} \{|(\na_a M^a_{AB}) u^Au^B| 
           + 2|I_Au^A|\} d \tilde{t} \nonumber \\
    %
    &\leq& \int^t_0 \int_{\Sigma_{\tilde{t}}} \{|C H_{AB} u^Au^B| \nonumber \\
          & + & 2\sqrt{|I_AI_B H^{AB}|}\sqrt{|H_{AB}u^Au^B|}\} d \tilde{t} \nonumber \\
    %
    &\leq& \int^t_0 [(C+1)E(u^A,\tilde{t}) + D(\tilde{t})] d \tilde{t}, \nonumber
\end{eqnarray}
%
where in the first member of the second inequality we have used the Schwartz inequality (see exercise) and have defined, 

\beq 
    C^2 := \sup_{\Omega} {(\na_aM^a_{AB})(\na_bM^b_{CD})H^{AC}H^{BD} }. 
\eeq 
In the third inequality we have used that $2ab \leq a^2 + b^2$ and have defined, 
\beq 
    D(t) := \int_{\Sigma_t} H^{AB}I_AI_B. 
\eeq 

\espa

\ejer: Let $H_{AB}$ be symmetric and positive definite, with inverse $H^{AB}$. Prove:

\noi a.) $S^{AB}S^{CD}H_{AC}H_{BD} \geq 0$, $(=0;iff;S^{AB}=0)$.

\noi b.) $|S_{AB}u^Au^B| \leq \sqrt{H^{AC}H^{BD}S_{AB}S_{CD}}; H_{AB}u^A u^B $. 

\espa

We will now make an important assumption that we will discuss in detail later: 
\espa

\noi{\bf we will assume from now on that $E(u^A,s) \geq 0 ;\forall; u^A$}. 
\espa

With this assumption, we can ignore this term in the previous inequality and obtain, \beq E(u^A,t) - E(u^A,0)
\leq \int^t_0 [(C+1)E(u^A,\tilde{t}) + D(\tilde{t})] d \tilde{t}. \eeq

Differentiating this integral inequality, we will obtain a maximum bound for the energy. 
Indeed, differentiating this expression and noting that the sign of the inequality is maintained, 
we obtain the following differential inequality, 
\beq 
\frac{d}{dt} E(u^A,t) \leq (1+C) E(u^A,t) + D(t), 
\eeq 
The differential equality has as a solution (using the method of variation of constants, section $5.2$), 
\beq 
Y(t) = e^{(1+C)t)};[Y(0) + \int_0^t e^{-(1+C)\tilde{t}}D(\tilde{t}) d \tilde{t}]. 
\eeq 
Using now lemma \ref{lem4_1} we see that 
$E(u^A,t) \leq Y(t);\forall;t\geq 0$ if $E(u^A,0) = Y(0)$, 
and therefore we have that, 
\beq 
E(u^A,t) \leq e^{(1+C)t)};[E(u^A,0) + \int_0^t e^{-(1+C)\tilde{t}}D(\tilde{t}) d \tilde{t}]. 
\eeq

This inequality is extremely important, not only does it allow us to infer the uniqueness of solutions 
(as we will see below) but it also plays a fundamental role in proving the existence of solutions and 
achieving convergent and reliable numerical algorithms.

%%%%%%%%%%%%%%%%%%%%%%%%%%%%%%%%%%%%%%%%%%%%%%%%%%%%%%%%%%%%%%%%%%%%%%%%%%%%

\section{Uniqueness of solutions}

Using the inequality obtained earlier, we will prove the following theorem:

\bteo Let there be a symmetric-hyperbolic equation on a manifold $M$. 
Let $\Sigma_0$ be a surface given by the equation $\tau = 0$ such that $M^a_{AB}\na_a\tau$ is positive definite. 
Let $\Gamma$ be any region of $\Sigma_0$ and let $\Omega $ be a region such that 
$\Omega \cap \Sigma_0= \Gamma$ and such that $E(u^A,s) \geq 0 ;\forall u^A$. [See previous figure.] 
Then if $u^C$ and $\tilde{u}^C$ are two solutions that coincide in $\Gamma$, they coincide in all of $\Omega$. 
\eteo

\pru: Let $\delta^A := u^A - \tilde{u}^A$. Then $\delta^A$ satisfies, 
\beq 
M^a_{AB}\na_a \delta^A = 0. 
\eeq 
Therefore, we have an energy inequality for $\delta^C$ with $D(t) \equiv 0$ and also with $E(\delta^C,0)=0$ since the two solutions coincide in $\Gamma$. 
But then the inequality tells us that $E(\delta^C,t)=0$ for all $t$ and therefore $\delta^C=0$ in all of $\Omega$ thus proving the theorem $\spadesuit$

%%%%%%%%%%%%%%%%%%%%%%%%%%%%%%%%%%%%%%%%%%%%%%%%%%%%%%%%%%%%%%%%%%%%%%%%%%%%

\section{Domain of dependence}

The previous uniqueness theorem was based on the assumption that \beq E(u^C,s) = \int_S n_aM^a_{AB}u^Au^B \geq 0 \eeq and therefore it is important to determine what are the possible regions where this happens. In particular, given a region $\Gamma$, the largest region $\Omega$ where we have uniqueness of solutions with identical initial data in $\Gamma$ is called the {\bf domain of dependence of $\Gamma$}, it is the region that depends completely on the initial data given in $\Gamma$, that is, giving initial data in $\Gamma$ we can completely control the value of the solution at any point in its domain of dependence.

First, let's see that this domain of dependence is not empty. To do this, take $\Gamma$ compact in $\Sigma_0$ and consider $H_{AB} = t_aM^a_{AB}$. Now let $\sigma = \tau - \delta \xi$ with $\xi$ a smooth function in a neighborhood of $\Gamma$ positive inside this set and negative in $\Sigma_0 - \Gamma$ (that is, it vanishes on its boundary) and $\delta$ a real number that we will assume small. 
[See figure \ref{fig:13_6c}.]

\espa 
\begin{figure}[htbp] 
    \begin{center} \resizebox{7cm}{!}{\myinput{Figure/m13_6c}} 
        \caption{Bubble-shaped region} 
        \label{fig:13_6c} 
    \end{center}
\end{figure}

At each point $p \in \Gamma$ $H_{AB}$ is a positive definite metric and therefore, as the set of positive definite metrics is open in the space of all symmetric tensors, 
given any other covector $w_a$ there will exist $\eps > 0$ small enough such that $(t_a + \eps w_a)M^a_{AB} = H_{AB} + \eps w_aM^a{AB}$ is also positive definite. 
Since $\Gamma$ is compact given a $w_a$ in it there will be a minimum and positive $\eps$ such that the tensor defined above will be positive~\footnote{To see this consider the map between 
$(B_1 \times \Gamma) \times (B_1 \times \Gamma) \times \Gamma \to \re$ given by, $w_a(p)M^a_{AB}(p)u^A(p)u^B(p) $ where $B_1(p) = {u^A(p)|H_{AB}(p)u^Au^B = 1}$. 
This is a continuous map and its domain is compact therefore it has a maximum, $m < \infty$. We can then take $0 < \eps < 1/m$.} 
By the same continuity argument, there will be a compact region around $\Gamma$ and a $\eps > 0$, a little smaller than the previous one such that there $\na_a\sigma M^a_{AB}$ will be positive definite. We have thus managed to have a region $\Omega$ between the level surfaces $\tau = 0$ and $\sigma = 0$, that is, a small {\sl bubble} where the integral of the outgoing energy through $S = {p\in M|\sigma(p) = 0}$ is positive for any solution $u^A$.

How large can we make this bubble, that is, how much more can we tilt the surface $S$ and still maintain positivity? 
This question has a lot to do with the following: how much can we tilt $t_a$ at each point and still have positivity of 
$t_aM^a_{AB}$ at that point?

First, note that if $t_aM^a_{AB}$ is positive then $\alpha t_aM^a_{AB}, \alpha >0 $, is also positive, so the set of covectors for which we have positivity forms a cone. This also ensures that not all covectors are in this cone, since if $t_a$ is in it, $(-t_a)$ is not.

Second, note that if $t_a$ and $\tilde{t}_a$ are in the cone, [that is, each of them gives a positive definite metric] then all covectors of the form, $\alpha t_a + (1-\alpha)\tilde{t}a, ;\alpha \in [0,1]$ are also in it, since $(\alpha t_aM^a{AB} + (1-\alpha)\tilde{t}aM^a{AB})u^A u^B$ is positive if the coefficients of $\alpha$ and $(1-\alpha)$ are. That is, the set of covectors that give positive definite metrics form a convex cone in $T_p^*$, at each point $p \in M$. This cone is called the {\bf characteristic cone}. 
[See figure \ref{fig:13_6b}.]

\espa % %\fig{5cm}{Characteristic cone} % 
\begin{figure}[htbp] 
    \begin{center} 
        \resizebox{7cm}{!}{\myinput{Figure/m13_6b}} 
        \caption{Characteristic cone} 
        \label{fig:13_6b} 
    \end{center}
\end{figure}

What happens to the covectors on the boundary of this cone? There the condition of positive definiteness must fail, that is, given a covector $t_a$ on this boundary there will be some $u^A$ such that $t_aM^a_{AB}u^Au^B = 0$. This implies that there the rank of $t_aM^a_{AB}$ ceases to be maximum.

\subsection{Construction of a characteristic surface}

Now we will construct the boundary surface of the maximum domain of dependence. 
From a region $\Gamma$ given by ${ q \in \Sigma_0 ;|;\sigma_0(p) = 0}$ we will construct a surface $S$ 
such that its normal at each of its points belongs to the boundary of the characteristic cone. 
This surface will be given by a function $\sigma = 0$ such that $\sigma|{\Sigma_0} = \sigma_0$. 
To find the equation that this function must satisfy, it is convenient to introduce an appropriate coordinate system, 
the same one we used in our classification of partial differential equations. 
One of these coordinates will be $t = \tau$ and we will call the others $x^i$, also for convenience we will define 
$\sigma_t=\derp{\sigma}{t}$ and $\sigma_i=\derp{\sigma}{x^i}$. With these coordinates we obtain that, 
\begin{eqnarray} 
    \na_a\sigma M^a{AB} &=& \sigma_t M^t_{AB} + \sum_i \sigma_i M^i_{AB} \nonumber \\ 
    &=& \sigma_t H_{AB} + \sum_i \sigma_i M^i_{AB}. 
\end{eqnarray} 
Multiplying by the inverse of $H_{AB}$, $H^{AB}$, we obtain 

\beq 
H^{CA}\na_a\sigma M^a_{AB} = \sigma_t \delta^C{}_B
\sum_i \sigma_i H^{CA}M^i_{AB}, 
\eeq 
which is an operator, that is, a linear map from a vector space to itself, and therefore we can take its determinant, obtaining, 
\beq 
det\left(\sigma_t \delta^C{}_B
\sum_i \sigma_i H^{CA}M^i_{AB}\right) = 0, 
\eeq since the determinant of $H^{CA}\na_a\sigma M^a_{AB}$ vanishes because we have assumed that the rank of 
$\na_a\sigma M^a_{AB}$ ceased to be maximum. 
For each fixed value of the spatial derivatives $\sigma_i$ this equation will generally have $n$ real solutions (roots), 
$\sigma_t$, (the eigenvalues of the operator $\sum_i \sigma_i H^{CA}M^i_{AB}$), of all of them we will take the one that gives us the boundary of the smallest cone containing $t_a$, that is, the largest root. We will thus have for this root an equation of the form: 

\beq 
\sigma_t + H(\sigma_i,x^i,t) = 0. \label{eiko} 
\eeq 

The function $H$ has a very important property, note that if $(\sigma_t,\sigma_i)$ is a solution so is $(\alpha \sigma_t,\alpha \sigma_i)$ and therefore $H$ must be 
{\bf homogeneous of first degree}, that is $ H(\alpha \sigma_i,x^i,t) = \alpha H(\sigma_i,x^i,t)$. 
These equations, with $H$ homogeneous of first degree are called {\bf eikonal equations} and are particular cases of the Hamilton--Jacobi equation studied in mechanics.
This type of equations can be solved by transforming them into an equivalent problem in ordinary derivatives derived from a Hamiltonian. 
Indeed, consider the system of ordinary Hamiltonian equations: 

\begin{eqnarray} 
    \derc{x^i}{t} &=& \derp{H}{p_i} \nonumber \\ 
    \derc{p_i}{t} &=& -\derp{H}{x^i}, \label{hamil} 
\end{eqnarray} 
where $H(p_i,x^i,t) := H(\sigma_i,x^i,t)|{\sigma_i=p_i}$. 
Integrating this system with initial conditions: 
\begin{eqnarray} 
    x^i(0) &=& x^i_0 \nonumber \\ 
    p_i(0) &=& \sigma{0i}, \label{condii} 
\end{eqnarray} 
and then restricting the integral curves obtained in the phase space $(x^i,p_j)$ to the configuration space $x^i$, we obtain a series of curves in our manifold 
emanating from the surface $\Gamma$. We will call these curves, {\bf characteristic curves} since they have the important property that along them the function 
$\sigma$ we are looking for is constant! To prove this first note that with the chosen initial conditions 
$p_i(t) = \sigma_i(x^j(t),t);\forall;t$. 
But taking the derivative with respect to $x^i$ of equation \ref{eiko} we obtain: 

\begin{eqnarray} 
    \frac{\partial\sigma_i}{\partial t} &=& \derssp{\sigma}{t}{x^i} \nonumber \\ 
    &=& \derssp{\sigma}{x^i}{t} \nonumber \\ 
    &=& -\sum_{j}\derp{H}{\sigma_j} (\sigma_k,x^k,t) \derp{\sigma_i}{x^j} - \derp{H}{x^i}(\sigma_k,x^k,t), 
    \label{deri} 
\end{eqnarray} 
%
and therefore 
\begin{eqnarray} 
    \derc{(p_i - \sigma_i)}{t} &=& - \derp{H}{x^i}(p_k,x^k,t)
    \sum_{j}\derp{\sigma_i}{x^j} \derc{x^j}{t} - \derp{\sigma_i}{t} \nonumber \\ 
    &=& - \derp{H}{x^i}(p_k,x^k,t) \sum_{j}\derp{\sigma_i}{x^j} \derp{H}{p_j}(p_k,x^k,t) \nonumber \\ 
    &+& \sum_{j}\derp{H}{\sigma_j} (\sigma_k,x^k,t) \derp{\sigma_i}{x^j} + \derp{H}{x^i}(\sigma_k,x^k,t) \nonumber \\
    &=& - \derp{H}{x^i}(p_k,x^k,t) + \derp{H}{x^i}(\sigma_k,x^k,t) \nonumber \\ 
    &-& \sum_{j}\derp{\sigma_i}{x^j} \left( \derp{H}{p_j}(p_k,x^k,t)
    \derp{H}{\sigma_j}(\sigma_k,x^k,t) \right), 
\end{eqnarray}
%
where in the second equality we have used equations \ref{hamil} and \ref{deri}. 
This last equation, with the chosen initial conditions has a trivial solution, but the uniqueness theorem of solutions of ordinary equations guarantees that this is the only one and therefore the sought equality. 
Therefore from now on we should not distinguish in the argument of $H$ whether it is evaluated in $\sigma_i$ or in $p_i$.
But then note that the derivative along a characteristic curve of $\sigma$ is, 

\begin{eqnarray} 
    \derc{\sigma}{t} &=& \sum_i \sigma_i\derc{x^i}{t} + \sigma_t \nonumber \\ 
    &=& \sum_i \derp{H}{\sigma_i} \sigma_i - H(\sigma_k,x^k,t) \nonumber \\ 
    &=& 0, 
\end{eqnarray} 
%
where in the last equality we have used that since $H$ is homogeneous of first degree 
in $\sigma_i$ the equality $ H(\sigma_k,x^k,t) = \sum_i \derp{H}{\sigma_i} \sigma_i$ holds.

We have thus demonstrated that $\sigma$ will be constant along the integral lines of equation \ref{hamil} with initial conditions given by \ref{condii}. 
Therefore we know $S$, this will be the surface ruled by the characteristic curves emanating from $\partial \Gamma$~\footnote{In general the characteristic curves intersect 
each other and therefore even giving a region $\Gamma$ with smooth boundary after a certain time the ruled surface will develop singularities and $\sigma$ will be multivalued. 
However these singularities are perfectly well known and it is known how to discard regions until obtaining domains of dependence with continuous boundaries.}. 
[See figure \ref{fig:13_7b}.]

\espa % %\fig{5cm}{Construction of $S$ and a singularity in $S$} % 
\begin{figure}[htbp] 
    \begin{center} \resizebox{9cm}{!}{\myinput{Figure/m13_7b}} 
        \caption{Construction of $S$ and a singularity in $S$} 
        \label{fig:13_7b} 
    \end{center}
\end{figure}

\subsection{Domain of dependence, examples}

Next we give a couple of examples of the construction of characteristic curves and determination of the domains of dependence. \espa

\ejem: {\bf Fluids in one dimension}

Consider a fluid with average density $\rho_0$, moving with average velocity $v_0$ and with an equation of state for the pressure as a function of density $p=p(\rho)$.
We are interested in small fluctuations of these quantities around the equilibrium state $(\rho_0,v_0)$, that is, in the theory of sound in this fluid. 
In this case $u^A = (\rho,v)$ will be these fluctuations and the fluid equations are 

\begin{eqnarray} 
    \derp{\rho}{t} - v_0 \derp{\rho}{x} - \rho_0 \derp{v}{x} &=& 0 \nonumber \\ 
    \derp{v}{t} - v_0 \derp{v}{x} - \frac{1}{\rho_0}\derp{p}{x} &=& 0. 
\end{eqnarray} If $c^2 := \derc{p}{\rho}|{\rho_0} > 0$, 
%
($c$ is the speed of sound in the medium) then the system is symmetric--hyperbolic with $M^a{AB}$ obtained by rewriting the equations as: 
\beq 
    \left( \begin{array}{cc} c^2 & 0 \\ 0 & \rho^2_0 
            \end{array} 
            \right) 
            \left( \begin{array}{c} \rho \\ v \end{array} \right)_t + \left( \begin{array}{cc} -c^2 v_0 & -c^2 \rho_0 \\ 
                - c^2 \rho_0 & -\rho^2_0 v_0 \end{array} \right) \left( \begin{array}{c} \rho \\ 
                    v \end{array} \right)_x = 0, 
\eeq 
%
where we have used that $\derp{p}{x} = \derp{p}{\rho} \derp{\rho}{x}$.

The determinant we must study is then: 
\beq 
det \left( \begin{array}{cc} \sigma_t c^2 - \sigma_x c^2 v_0
& -\sigma_x c^2 \rho_0 \\ & \\ - \sigma_x c^2 \rho_0
& \sigma_t \rho^2_0 - \sigma_x \rho v \end{array} \right), 
\eeq which has as roots, \beq \sigma_t = (v_0 \pm c) \sigma_x. 
\eeq 
Suppose $c > v_0$ (normal fluid, subsonic), and that $\Gamma = [0,1]$ with $\sigma_0 = x(x-1)$. At $x=0$ $\sigma_{0x} $ is negative and then the largest root is $v_0-c$ and the solution is $\sigma_- = t(c-v_0) - x$. At $x=1$ $\sigma_{0x} $ is positive and then the largest root is $v_0+c$ and the solution is $\sigma_- = t(v_0+c) - (x-1)$. [See figure.]

\espa % %\fig{5cm}{Domain of dependence of a fluid} % 
\begin{figure}[htbp] 
    \begin{center} 
        \resizebox{7cm}{!}{\myinput{Figure/m13_8b}} 
        \caption{Domain of dependence of a fluid} 
        \label{fig:13_8b} 
    \end{center}
\end{figure}

\espa

\ejem: {\bf Wave equation in $2+1$ dimensions.}

The equation is: 

\beq 
-\derssp{\phi}{t}{t} + \derssp{\phi}{x}{x} + \derssp{\phi}{y}{y} = \rho, \label{onda2d} 
\eeq 
and using 
$u^A = (\phi,\derp{\phi}{t},\derp{\phi}{x},\derp{\phi}{y})$ 
the system can be written as, 
{\small
\arraycolsep=2pt
%\begin{equation}
$$
\left(
      \begin{array}{cccc}
      1&0&0&0 \\
      0&1&0&0 \\
      0&0&1&0 \\
      0&0&0&1
      \end{array}
\right)      
\left(
      \begin{array}{c}
      u^1\\u^2\\u^3\\u^4
      \end{array}
\right)_t
-
\left(
      \begin{array}{cccc}
      0&0&0&0 \\
      0&0&1&0 \\
      0&1&0&0 \\
      0&0&0&0
      \end{array}
\right)      
\left(
      \begin{array}{c}
      u^1\\u^2\\u^3\\u^4
      \end{array}
\right)_x
-
\left(
      \begin{array}{cccc}
      0&0&0&0 \\
      0&0&0&1 \\
      0&0&0&0 \\
      0&1&0&0
      \end{array}
\right)      
\left(
      \begin{array}{c}
      u^1\\u^2\\u^3\\u^4
      \end{array}
\right)_y
= 
\left(
      \begin{array}{c}
     - u^2\\-\rho\\0\\0
      \end{array}
\right) 
\nonumber
%\end{equation}
$$
}
\arraycolsep=5pt
\espa

\ejer: Prove that if the initial data is such that $u^2=\derp{u^1}{x}$ and $u^3=\derp{u^1}{y}$, then $u^1$ satisfies \ref{onda2d}. \espa

The equations for $\sigma$ are: 
\beq 
det \left( \begin{array}{cccc} \sigma_t&0&0&0 \\ 0&\sigma_t&\sigma_x&\sigma_y\\ 0&\sigma_x&\sigma_t&0 \\ 0&\sigma_y&0&\sigma_t 
\end{array} \right) = (\sigma_t)^2[(\sigma_t)^2 - (\sigma_x)^2 -(\sigma_y)^2], 
\eeq 
that is 
\beq 
\sigma_t = \sqrt{(\sigma_x)^2 + (\sigma_y)^2} := - H(\sigma_x,\sigma_y). 
\eeq 

The Hamilton equations for this system are: 
\begin{eqnarray}
    \derc{x}{t} &=& \derp{H}{\sigma_x}
                 = \frac{-\sigma_x}{\sqrt{(\sigma_x)^2 
                    + (\sigma_y)^2}}  \nonumber \\
                & & \nonumber \\
    \derc{y}{t} &=& \derp{H}{\sigma_y} 
                 = \frac{-\sigma_y}{\sqrt{(\sigma_x)^2 
                    + (\sigma_y)^2}} \nonumber \\             
                & & \nonumber \\    
    \derc{\sigma_x}{t} &=& - \derp{H}{x} 
                 = 0  \nonumber \\
                & & \nonumber \\
    \derc{\sigma_y}{t} &=& - \derp{H}{y} 
                 = 0,
    \end{eqnarray}
    %
and their solutions are: 
\begin{eqnarray}
    x(t) &=&  \frac{-\sigma_{0x}}{\sqrt{(\sigma_{0x})^2 
                    + (\sigma_{0y})^2}} t + x_0    \nonumber \\
    y(t) &=& \frac{-\sigma_{0y}}{\sqrt{(\sigma_0x)^2 
                    + (\sigma_{0y})^2}} t + y_0.      
\end{eqnarray}      


%\end{document}

%%% Local Variables: 
%%% mode: latex 
%%% TeX-master: t 
%%% End:
